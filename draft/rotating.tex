\documentclass[preprint,aps,tightenlines,showkeys,nofootinbib,superscriptaddress,amsmath]{revtex4}
\usepackage{latexsym}
\usepackage[dvips]{graphicx}
%  Mu-In's macros
%\def\rref#1{(\ref{#1})}
\newcommand{\beq}{
\begin{eqnarray}}
  \newcommand{\eeq}{
\end{eqnarray}}
\newcommand{\bea}{
\begin{eqnarray*}}
  \newcommand{\eea}{
\end{eqnarray*}}
\newcommand{\eq}{eqnarray}
\newcommand{\bb}{\bibitem}
\newcommand{\al}{{\alpha}}
\newcommand{\be}{{\beta}}
\newcommand{\ci}{\cite}
\newcommand{\ga}{{\gamma}}
\newcommand{\Ga}{{\Gamma}}
\newcommand{\ep}{{\epsilon}}
\newcommand{\epi}{{\epsilon^{ij}}}
\newcommand{\bepi}{\bar{\epsilon}^{ij}}
\newcommand{\bep}{{\bf \epsilon}}
\newcommand{\de}{{\delta}}
\newcommand{\De}{\Delta}
\newcommand{\ka}{\kappa}
\newcommand{\tD}{\tilde{\Delta}}
\newcommand{\Th}{{\Theta}}
\newcommand{\bT}{{\bf \Theta}}
\newcommand{\la}{{\lambda}}
\newcommand{\La}{{\Lambda}}
\newcommand{\m}{{\mu}}
\newcommand{\n}{{\nu}}
\newcommand{\si}{{\sigma}}
\newcommand{\Si}{{\Sigma}}
\newcommand{\om}{{\omega}}
\newcommand{\Om}{{\Omega}}
\newcommand{\pa}{{\partial}}
\newcommand{\no}{{\nonumber}}
\newcommand{\f}{\frac}
\newcommand{\ra}{\rightarrow}
\newcommand{\lra}{\leftrightarrow}
\newcommand{\Diff}{\hbox{\scriptsize Diff}}
\newcommand{\eff}{\hbox{\scriptsize eff}}
\newcommand{\tot}{\hbox{\scriptsize tot}}
\newcommand{\stat}{\hbox{\scriptsize stat}}
\newcommand{\exact}{\hbox{\scriptsize exact}}
\newcommand{\mmin}{\hbox{\scriptsize min}}
\newcommand{\new}{\hbox{\scriptsize new}}
\newcommand{\mmax}{\hbox{\scriptsize max}}
\newcommand{\AdS}{\hbox{\scriptsize AdS}}
\newcommand{\BTZ}{\hbox{\scriptsize BTZ}}
\newcommand{\HMTZ}{\hbox{\scriptsize HMTZ}}
\newcommand{\BH}{\hbox{\scriptsize BH}}
\newcommand{\Euc}{Euclidean }
\newcommand{\Sch}{Schwarzschild }
\newcommand{\ther}{thermodynamics }
\newcommand{\therl}{thermodynamical }
\newcommand{\mic}{micro-canonical }
\newcommand{\can}{canonical }
\newcommand{\lo}{logarithmic }
\newcommand{\temp}{temperature }
\newcommand{\cha}{characteristic}
\newcommand{\mln}{\mbox{ln}}
\newcommand{\hb}{\hat{\be}}
\newcommand{\hO}{\hat{\Om}}
\newcommand{\he}{\hat{\eta}}
\newcommand{\we}{\wedge}
\newcommand{\na}{\nabla}
\newcommand{\bn}{\bar{\nabla}}
\newcommand{\bg}{\bar{g}}
\newcommand{\fn}{\footnote}
\newcommand{\appr}{approximation}
\newcommand{\kI}{\kappa^{-1} \epi}
\newcommand{\GI}{(16 \pi G)^{-1} \epi}
\newcommand{\G}{\f{1}{16 \pi G} \epi}
\newcommand{\GG}{\f{2}{16 \pi G} \epi}
\newcommand{\mG}{\f{2 \m}{16 \pi G} \epi}
\newcommand{\dd}{\de^2 (x-y)}
\newcommand{\asy}{asymptotically}
\newcommand{\Asy}{Asymptotically}
\newcommand{\SchAdS}{Schwartzshild-AdS}
\newcommand{\SchdS}{Schwartzshild-dS}
\newcommand{\SchAAdS}{Schwartzshild-(A)dS}
\newcommand{\Ho}{Ho\v{r}ava}
\newcommand{\diff}{diffeomorphism}
\newcommand{\DiffF}{${\it Diff}_{\cal F}$}
%%%%%%%%%%%%%%% End of Private Macros %%%%%%%%%%%%%%%%%%%%%%%%%%

\begin{document}

\preprint{arXiv:2402.xxxx [hep-th]}

\title{Rotating Black Holes with Illegitimate transformations}

\author{Deniz O. Devecio\u{g}lu \footnote{E-mail address: dodeve@gmail.com}}
\affiliation{\orgname{Daegu Gyeongbuk Institute of Science and
  Technology (DGIST)},\\ \orgaddress{\street{333 Techno Jungang-daero,
Dalseong-gun}, \city{Daegu} \postcode{42988}, \country{Republic of Korea}} }

\author{Mu-In Park \footnote{E-mail address: muinpark@gmail.com,
Corresponding author}}
\affiliation{ Center for Quantum Spacetime, Sogang University,
Seoul, 121-742, Korea }
\date{\today}

\begin{abstract}
  this is the abstract

\end{abstract}

%\pacs{04.20.Jb, 04.20.Dw, 04.60.Kz, 04.60.-m, 04.70.Dy }
\keywords{Horava gravity, Rotating black hole solutions, Black hole
thermodynamics, Gravitational waves, Lorentz violations}

\maketitle

\newpage

\section{Introduction}

\section{Illegitimate coordinate transformations}
We start with giving examples of 'illegitimate' transformation discussed in
\cite{Martinez:1999qi}. The transformations are illegitimate in the
sense that does not preserve the properties of the coordinates they
transform, e.g. periodicity or the range of the variable.

This way of generating solutions from known ones by a coordinate
transformation was used before in number of papers
\cite{Deser:1985pk,Deser:1983tn} but as pointed out in
\cite{Martinez:1999qi} there is no rationale behind it.

The easiest example we can discuss is from \cite{Deser:1985pk} where a
stationary and axially symmetric metric is given by
\begin{equation}
  ds^{2}= dr^{2}+f^{2}(2)d \theta^{2}+2g_{02}(r)dt d\theta +g_{00}(r)dt^{2},
\end{equation}
with the components
\begin{eqnarray}
  g_{02}& = &A(1-AF)+Fr^{2},\\
  -g_{00}& =&(1-AF)^{2}-F^{2}r^{2},\\
  f^{2}(r)&=&r^{2}-A^{2}.
\end{eqnarray}
This metric can be cast in Minkowski form under the following transformations
\begin{eqnarray}
  \Theta &=&\theta+Ft,\label{tran1}\\
  T &=&t-A\Theta. \label{tran2}
\end{eqnarray}
These transformations can be called illegitimate as the angular
variable periodicity is not preserved. Next, we can write down the
Jacobian for these transformations (transformation from
$(t,r,\theta)\rightarrow (T,r,\Theta))$
\begin{eqnarray}
  \frac{\partial x^{\alpha} }{\partial x^{\prime\mu} }=
  \begin{pmatrix}
    1 & 0 & A \\
    0 & 1 & 0 \\
    -F & 0 & (1-AF)
  \end{pmatrix}. \label{jacobflat}
\end{eqnarray}
For the sake of simplicity we only give the transformation for the $g_{TT}$ term
\begin{eqnarray}
  g_{TT}=g_{tt}+2 \frac{\partial\theta}{\partial T} g_{\theta
  t}+\left(\frac{\partial\theta}{\partial T}\right)^{2}g_{\theta \theta }=-1,
\end{eqnarray}
going over all the metric components it is easy to show that the
transformed metric is Minkowski
\begin{equation}
  ds^{2}=-dT^{2}+ dr^{2}+r^{2}d\Theta^{2}.
\end{equation}
\section{Illegitimate transformations (non-coordinate basis)}
The simplest example for our case comes from 3 dimensional gravity
solution called $\text{WAdS}_{3}$ which is a deformation of
$\text{AdS}_3$. The deformations can be squashing or streching of
$\text{AdS}_3$. There are several representations of
$\text{WAdS}_{3}$ and we will use the following simple form for our purposes
the one we use is from \cite{Giribet:2015lfa}.
\begin{equation}
  ds^{2}=\dfrac{l^{2}_K}{4}\left(-\cosh\x d\tau^{2}+dx^{2}+K(dy+\sinh
  x d\tau )^{2}   \right)\label{warpedads}
\end{equation}
where we have introduced
\begin{eqnarray}
  K=\frac{4\nu^{2}}{(\nu ^{2}+3 )},\quad l_{K}^{2}=\frac{4l^{2}
  }{(\nu ^{2}+3 )},
\end{eqnarray}
$\nu =1$ corresponds to locally $AdS_3$ geometries and spaces with
$\nu ^{2}>1$ is streched $\nu ^{2}<1$ are called squashed
deformations of $AdS_3$.

Moreover, unlike $AdS_3$, the warped versions are not solutions to
Einstein gravity in 3 dimensions. To obtain them, one needs to couple
matter or consider models like Topologically Massive Gravity (TMG) or New
Massive Gravity (NMG) (higher order derivative theories). For example
(\ref{warpedads}) is a solution to TMG field equations
\begin{equation}
  R_{\alpha\beta}+\frac{1}{\mu } C_{\alpha \beta }+\Lambda g_{\mu  \nu }=0
\end{equation}
with $C_{\mu \nu }$ being the Cotton tensor
\begin{eqnarray}
  C^{\sigma \rho }&=\varepsilon^{\alpha \beta (\sigma }\nabla_{\alpha
  }R^{\rho) }{}_{\beta },\nonumber\\
  \mu &=\dfrac{3\nu}{L},\quad\Lambda=-\dfrac{1}{L^{2} }
\end{eqnarray}

Now, for the transformation between $\nu =1$ and $\nu \neq 1$ case of
(\ref{warpedads}), consider two coordinate systems with
$(\tau,x,y)$ and $(T,X,Y)$. The transformation between them will take
the the $AdS_3$ metric
\begin{equation}
  ds^{2}=\frac{l^{2}}{4}\left(-\cosh x d\tau^{2}+dx^{2}+(dy+\sinh
  x d\tau )^{2} \right),
\end{equation}
to the one given in (\ref{warpedads})
\begin{equation}
  ds^{2}=\dfrac{l^{2}_K}{4}\left(-\cosh X dT^{2}+dX^{2}+K(dY+\sinh
  X dT )^{2}   \right)\label{warpedads}.
\end{equation}
Note that in this case two spacetimes we consider are not even the
solution of same theory. The Jacobian of transformation is given by
\begin{eqnarray}
  \frac{\partial x^{\alpha} }{\partial x^{\prime\mu} }=
  \begin{pmatrix}
    \dfrac{2}{\sqrt{3+\nu ^{2} } } & 0 & 0 \\
    0 & \dfrac{\sqrt{2}}{(3+\nu ^{2} ) ^{1 / 4}}    & 0 \\
    -\frac{2(3+\nu ^{2}-2 \nu \sqrt{3+\nu ^{2} }  )}{(3+\nu ^{2} )
    ^{3/2 }}\sinh \left(\frac{(3+\nu ^{2})^{1 / 4}}{\sqrt{2}}X\right)  & 0 &
    \dfrac{4 \nu }{(3+\nu ^{2})}
  \end{pmatrix}. \label{jacobwads}
\end{eqnarray}
The metric components are
\begin{eqnarray}
  g_{TT}&=&\left(\dfrac{\partial\tau}{\partial T}\right) ^{2} g_{\tau
  \tau }+ 2 \dfrac{\partial \tau}{\partial T}\dfrac{\partial
  y}{\partial T}g_{\tau y}+\left(\dfrac{\partial y}{\partial
  T}\right)^{2} g_{yy},\\
  g_{XX}&=& \left(\dfrac{\partial x}{\partial X}\right)^{2} g_{x x},\\
  g_{TY}&=& \dfrac{\partial \tau}{\partial T}\dfrac{\partial
  y}{\partial Y}g_{\tau y}+\dfrac{\partial y}{\partial
  T}\dfrac{\partial y}{\partial Y}g_{yy},\\
  g_{YY}&=&\left(\dfrac{\partial y}{\partial Y}\right)^{2}.
\end{eqnarray}
After this transformation we are close to the form want (\ref{warpedads}).
The metric now reads
\begin{eqnarray}
  ds^{2}&=&-\dfrac{L ^{2} }{(\nu ^{2}+3)}\left(\cosh \left(\frac{(\nu
    ^{2}+3)^{1 / 4}}{\sqrt{2}}X\right)^{2}-\dfrac{4 \nu ^{2} }{(\nu
    ^{2}+ 3 )}\sinh \left(\frac{(3+\nu
  ^{2})^{1 / 4}}{\sqrt{2}} X \right)^{2}  \right)dT^{2}\\
  &+& \dfrac{L^{2} }{2\sqrt{\nu ^{2}+3 } }dX^{2}+\dfrac{4 L^{2} \nu
  ^{2} }{(\nu ^{2}+3 )^{2} }dY^{2}+\drac{8L^{2}\nu ^{2}\sinh \left(\frac{(3+\nu
  ^{2})^{1 / 4}}{\sqrt{2}} X \right)  }dT dY.
\end{eqnarray}
With a final scaling $dX=\sqrt{2} d\bar X / (\nu ^{2}+3 )^{1 / 4} $
we have the $\text{WAdS}_{3}$ in the form of (\ref{warpedads}).

The Jacobian (\ref{jacobwads}) of the transformation has a non-zero
determinant so its a locally valid coordinate transformation. However,
unlike the first example (\ref{jacobflat}) the differentials defined
by each row is not exact i.e. it is not possible to write all transformations as
functions in (\ref{tran1}), (\ref{tran2}). So this type of
transformations seems like more general than the illegitimate ones.

\subsection{Gibbons $AdS_3$ to Horava}
In this section we will consider the transformation from Gibbons form
of GR solution to the Horava-type solution. We start with the Gibbons
solution given in \cite{Gibb:2004} (with $\lambda = 0$)
\begin{eqnarray}
  ds^{2} &= & \left(  -1+\left(\frac{2Mr}{a^{2}+r^{2}-a^{2}\mu ^{2}
  }\right)\right)dt^2
  -\frac{4aMr \mu^{2}}{a^{2}+r^{2}-a^2 \mu^2   }dt d
  \varphi+\frac{a^{2}+r^{2}-a^{2}\mu ^{2}    }{a^{2}-2Mr+r^{2}  }dr^2
  \nonumber\\
  & +& \left(a^{2} - \frac{r^{2}}{\mu ^{2}-1 } \right)d \theta ^{2}
  +\frac{a^2(a^2+r(r-2M))\mu ^{4}-(r^2+a^2)^2 \mu ^2}{a^2(\mu
  ^2-1)-r^2 }d \varphi ^2
\end{eqnarray}
Then we consider the following transformation with the unknown
function $H(r,\mu )$
\begin{eqnarray}
  d \tau  = dt,\quad dR=dr,\quad d \tilde \mu  = d \mu, \quad d
  \tilde \varphi -H(r,\mu )d \tau = d \varphi,
\end{eqnarray}
where $\tau ,R,\tilde \mu , \tilde \varphi $ are the new coordinates.
The new components of the metric
function can be found solving the following set
\begin{eqnarray}
  g_{t t}-2g_{t \varphi }H(r,\varphi )+g_{\varphi \varphi
  }H(r,\varphi )^{2}-g_{\tau \tau }&=&0,\\
  g_{t \varphi }-g_{\varphi  \varphi }-g_{\tau \tilde \varphi }&=&0,
\end{eqnarray}
in which the solution is given by (note that the rest of the
components does not transform)
\begin{eqnarray}
  g_{\tau \tau} &=& \frac{{\mu }^2 H \left(H
      \left(\left(a^2+r^2\right)^2-a^2 {\mu }^2 \left(a^2+r (r-2
    M)\right)\right)+4 a M r\right)+a^2 \left({\mu }^2-1\right)+r (2
  M-r)}{r^2-a^2 \left({\mu }^2-1\right)},\\
  g_{\tau \tilde\varphi} &=&\frac{\mu ^2 \left(H
      \left(\left(a^2+r^2\right)^2-a^2 \mu ^2 \left(a^2+r (r-2
  M)\right)\right)+2 a M r\right)}{a^2 \left(\mu ^2-1\right)-r^2}.
\end{eqnarray}
In terms of ADM variables this only introduces a change in shift function
\begin{eqnarray}
  N_{\phi} = \frac{\mu ^{2}(2aMr+((a^{2}+r^2 )^{2}-a^{2}(a^{2}+(r-2M)
  )\mu ^{2}   )H) }{a^{2}(\mu^{2}-1 ) r^{2} }.
\end{eqnarray}
Using the new metric components in Horava's equations we will have
PDEs for the function $H(r,\mu )$ which are discussed in mathematica files.

\section{Hocava}

\begin{\eq}
  S_g &= & \int_{{\bf R} \times \Si_t} dt d^3 x
  \sqrt{g}N\left[\frac{1}{\kappa}\left(K_{ij}K^{ij}-\lambda
  K^2\right)+\xi R^{}-2 \La+\f{\sigma}{2} a_i a^i \right]\ ,
  \label{horava}
\end{\eq}
where
%\begin{\eq}
$
K_{ij}=({2N})^{-1}\left(\dot{g}_{ij}-\nabla_i
N_j-\nabla_jN_i\right)
$
% \end{\eq}
is the extrinsic curvature [the overdot $(\dot{})$ denotes the time
  derivative and $\nabla_i$ is the covariant derivatives for the
induced metric $g_{ij}$ on the time-slicing hypersurface $\Si_t$] and
$R$ is the {\it three}-curvature in the ADM metric
\begin{\eq}
  \label{metric}
  ds^2=-N^2 c^2 dt^2+g_{ij}\left(dx^i+N^i dt\right)\left(dx^j+N^j
  dt\right),
\end{\eq}
respectively. (Hereafter, we shall use the unit $c=1$, unless
stated otherwise.)
The last term in the gravity action (\ref{horava}) is introduced for
completeness, with the proper acceleration $a_i=\nabla_i ln N$
\cite{Blas:2009}.

\section*{Acknowledgments}

Thanking part.

\appendix

\section{Explicit expressions of $\widehat{\rho}$ and $\widehat{p_a}$ in
  ${T^{\m}}_{\n (\it{eff})}$
  %the effective energy-momentum tensor components
}

%%%%%%%%%% References %%%%%%%%%%%%%%%%%%%%%%%%%
\newcommand{\J}[4]{#1 {\bf #2} #3 (#4)}
\newcommand{\andJ}[3]{{\bf #1} (#2) #3}
\newcommand{\AP}{Ann. Phys. (N.Y.)}
\newcommand{\MPL}{Mod. Phys. Lett.}
\newcommand{\NP}{Nucl. Phys.}
\newcommand{\PL}{Phys. Lett.}
\newcommand{\PR}{Phys. Rev. D}
\newcommand{\PRL}{Phys. Rev. Lett.}
\newcommand{\PTP}{Prog. Theor. Phys.}
\newcommand{\hep}[1]{ hep-th/{#1}}
\newcommand{\hepp}[1]{ hep-ph/{#1}}
\newcommand{\hepg}[1]{ gr-qc/{#1}}
\newcommand{\bi}{ \bibitem}
%%%%%%%%%%%%%%%%%%%%%%%%%%%%%%%%%%%%%%%%%%%%%%%

\begin{thebibliography}{999}
  %\cite{Giribet:2015lfa}
  \bibitem{Giribet:2015lfa}
  G.~Giribet and M.~Tsoukalas,
  %``Warped-AdS3 black holes with scalar halo,''
  Phys. Rev. D \textbf{92}, no.6, 064027 (2015)
  doi:10.1103/PhysRevD.92.064027
  [arXiv:1506.05336 [gr-qc]].
  %17 citations counted in INSPIRE as of 03 Jul 2025
  %\cite{Deser:1983tn}
  \bibitem{Deser:1983tn}
  S.~Deser, R.~Jackiw and G.~'t Hooft,
  %``Three-Dimensional Einstein Gravity: Dynamics of Flat Space,''
  Annals Phys. \textbf{152}, 220 (1984)
  doi:10.1016/0003-4916(84)90085-X
  %1197 citations counted in INSPIRE as of 03 Jul 2025
  %\cite{Deser:1985pk}
  \bibitem{Deser:1985pk}
  S.~Deser and P.~O.~Mazur,
  %``Static Solutions in $D=3$ Einstein-maxwell Theory,''
  Class. Quant. Grav. \textbf{2}, L51 (1985)
  doi:10.1088/0264-9381/2/3/003
  %46 citations counted in INSPIRE as of 03 Jul 2025
  %\cite{Martinez:1999qi}
  \bibitem{Martinez:1999qi}
  C.~Martinez, C.~Teitelboim and J.~Zanelli,
  %``Charged rotating black hole in three space-time dimensions,''
  Phys. Rev. D \textbf{61}, 104013 (2000)
  doi:10.1103/PhysRevD.61.104013
  [arXiv:hep-th/9912259 [hep-th]].
  %324 citations counted in INSPIRE as of 03 Jul 2025

  \bibitem{Kerr:1963}
  R.~P.~Kerr,
  %``Gravitational field of a spinning mass as an example of
  % algebraically special metrics,''
  Phys.\ Rev.\ Lett.\  {\bf 11}, 237 (1963).
  % doi:10.1103/PhysRevLett.11.237

  %\cite{Chandrasekhar:1985kt}
  \bibitem{Chan:1985}
  S.~Chandrasekhar,
  ``The mathematical theory of black holes,'' (Oxford Univ.
    %ersity
  Press, 1983).
  %348 citations counted in INSPIRE as of 12 Jan 2024

  %\cite{Gibbons:2004uw}
  \bibitem{Gibb:2004}

  G.~W.~Gibbons \textit{et al.},
  %H.~Lu, D.~N.~Page and C.~N.~Pope,
  %``The General Kerr-de Sitter metrics in all dimensions,''
  J. Geom. Phys. \textbf{53}, 49
  %-73
  (2005), Appendix {\bf E}.
  %doi:10.1016/j.geomphys.2004.05.001
  %[arXiv:hep-th/0404008 [hep-th]].
  %424 citations counted in INSPIRE as of 12 Jan 2024

\end{thebibliography}

\end{document}
