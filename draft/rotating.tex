\documentclass[preprint,aps,tightenlines,showkeys,nofootinbib,superscriptaddress]{revtex4}
\usepackage{latexsym}
\usepackage[dvips]{graphicx}
%  Mu-In's macros
%\def\rref#1{(\ref{#1})}
\newcommand{\beq}{
\begin{eqnarray}}
  \newcommand{\eeq}{
\end{eqnarray}}
\newcommand{\bea}{
\begin{eqnarray*}}
  \newcommand{\eea}{
\end{eqnarray*}}
\newcommand{\eq}{eqnarray}
\newcommand{\bb}{\bibitem}
\newcommand{\al}{{\alpha}}
\newcommand{\be}{{\beta}}
\newcommand{\ci}{\cite}
\newcommand{\ga}{{\gamma}}
\newcommand{\Ga}{{\Gamma}}
\newcommand{\ep}{{\epsilon}}
\newcommand{\epi}{{\epsilon^{ij}}}
\newcommand{\bepi}{\bar{\epsilon}^{ij}}
\newcommand{\bep}{{\bf \epsilon}}
\newcommand{\de}{{\delta}}
\newcommand{\De}{\Delta}
\newcommand{\ka}{\kappa}
\newcommand{\tD}{\tilde{\Delta}}
\newcommand{\Th}{{\Theta}}
\newcommand{\bT}{{\bf \Theta}}
\newcommand{\la}{{\lambda}}
\newcommand{\La}{{\Lambda}}
\newcommand{\m}{{\mu}}
\newcommand{\n}{{\nu}}
\newcommand{\si}{{\sigma}}
\newcommand{\Si}{{\Sigma}}
\newcommand{\om}{{\omega}}
\newcommand{\Om}{{\Omega}}
\newcommand{\pa}{{\partial}}
\newcommand{\no}{{\nonumber}}
\newcommand{\f}{\frac}
\newcommand{\ra}{\rightarrow}
\newcommand{\lra}{\leftrightarrow}
\newcommand{\Diff}{\hbox{\scriptsize Diff}}
\newcommand{\eff}{\hbox{\scriptsize eff}}
\newcommand{\tot}{\hbox{\scriptsize tot}}
\newcommand{\stat}{\hbox{\scriptsize stat}}
\newcommand{\exact}{\hbox{\scriptsize exact}}
\newcommand{\mmin}{\hbox{\scriptsize min}}
\newcommand{\new}{\hbox{\scriptsize new}}
\newcommand{\mmax}{\hbox{\scriptsize max}}
\newcommand{\AdS}{\hbox{\scriptsize AdS}}
\newcommand{\BTZ}{\hbox{\scriptsize BTZ}}
\newcommand{\HMTZ}{\hbox{\scriptsize HMTZ}}
\newcommand{\BH}{\hbox{\scriptsize BH}}
\newcommand{\Euc}{Euclidean }
\newcommand{\Sch}{Schwarzschild }
\newcommand{\ther}{thermodynamics }
\newcommand{\therl}{thermodynamical }
\newcommand{\mic}{micro-canonical }
\newcommand{\can}{canonical }
\newcommand{\lo}{logarithmic }
\newcommand{\temp}{temperature }
\newcommand{\cha}{characteristic}
\newcommand{\mln}{\mbox{ln}}
\newcommand{\hb}{\hat{\be}}
\newcommand{\hO}{\hat{\Om}}
\newcommand{\he}{\hat{\eta}}
\newcommand{\we}{\wedge}
\newcommand{\na}{\nabla}
\newcommand{\bn}{\bar{\nabla}}
\newcommand{\bg}{\bar{g}}
\newcommand{\fn}{\footnote}
\newcommand{\appr}{approximation}
\newcommand{\kI}{\kappa^{-1} \epi}
\newcommand{\GI}{(16 \pi G)^{-1} \epi}
\newcommand{\G}{\f{1}{16 \pi G} \epi}
\newcommand{\GG}{\f{2}{16 \pi G} \epi}
\newcommand{\mG}{\f{2 \m}{16 \pi G} \epi}
\newcommand{\dd}{\de^2 (x-y)}
\newcommand{\asy}{asymptotically}
\newcommand{\Asy}{Asymptotically}
\newcommand{\SchAdS}{Schwartzshild-AdS}
\newcommand{\SchdS}{Schwartzshild-dS}
\newcommand{\SchAAdS}{Schwartzshild-(A)dS}
\newcommand{\Ho}{Ho\v{r}ava}
\newcommand{\diff}{diffeomorphism}
\newcommand{\DiffF}{${\it Diff}_{\cal F}$}
%%%%%%%%%%%%%%% End of Private Macros %%%%%%%%%%%%%%%%%%%%%%%%%%

\begin{document}

\preprint{arXiv:2402.xxxx [hep-th]}

\title{Rotating Black Holes with Illegitimate transformations}

\author{Deniz O. Devecio\u{g}lu \footnote{E-mail address: dodeve@gmail.com}}
\affiliation{\orgname{Daegu Gyeongbuk Institute of Science and
  Technology (DGIST)},\\ \orgaddress{\street{333 Techno Jungang-daero,
Dalseong-gun}, \city{Daegu} \postcode{42988}, \country{Republic of Korea}} }

\author{Mu-In Park \footnote{E-mail address: muinpark@gmail.com,
Corresponding author}}
\affiliation{ Center for Quantum Spacetime, Sogang University,
Seoul, 121-742, Korea }
\date{\today}

\begin{abstract}
  this is the abstract

\end{abstract}

%\pacs{04.20.Jb, 04.20.Dw, 04.60.Kz, 04.60.-m, 04.70.Dy }
\keywords{Horava gravity, Rotating black hole solutions, Black hole
thermodynamics, Gravitational waves, Lorentz violations}

\maketitle

\newpage

\section{Introduction}
So in this section there will be an introduction phase however it
will take time to write.
\begin{itemize}
  \item first one
  \item second one
\end{itemize}

\section{Illegitimate transformations}
In this section we will consider the transformation from Gibbons form
of GR solution to the Horava-type solution. We start with the Gibbons
solution given in \cite{Gibb:2004} (with $\lambda = 0$)
\begin{eqnarray}
  ds^{2} &= & \left(  -1+\left(\frac{2Mr}{a^{2}+r^{2}-a^{2}\mu ^{2}
  }\right)\right)dt^2
  -\frac{4aMr \mu^{2}}{a^{2}+r^{2}-a^2 \mu^2   }dt d
  \varphi+\frac{a^{2}+r^{2}-a^{2}\mu ^{2}    }{a^{2}-2Mr+r^{2}  }dr^2
  \nonumber\\
  & +& \left(a^{2} - \frac{r^{2}}{\mu ^{2}-1 } \right)d \theta ^{2}
  +\frac{a^2(a^2+r(r-2M))\mu ^{4}-(r^2+a^2)^2 \mu ^2}{a^2(\mu
  ^2-1)-r^2 }d \varphi ^2
\end{eqnarray}
Then we consider the following transformation with the unknown
function $H(r,\mu )$
\begin{eqnarray}
  d \tau  = dt,\quad dR=dr,\quad d \tilde \mu  = d \mu, \quad d
  \tilde \varphi -H(r,\mu )d \tau = d \varphi,
\end{eqnarray}
where $\tau ,R,\tilde \mu , \tilde \varphi $ are the new coordinates.
The new components of the metric
function can be found solving the following set
\begin{eqnarray}
  g_{t t}-2g_{t \varphi }H(r,\varphi )+g_{\varphi \varphi
  }H(r,\varphi )^{2}-g_{\tau \tau }&=&0,\\
  g_{t \varphi }-g_{\varphi  \varphi }-g_{\tau \tilde \varphi }&=&0,
\end{eqnarray}
in which the solution is given by (note that the rest of the
components does not transform)
\begin{eqnarray}
  g_{\tau \tau} &=& \frac{{\mu }^2 H \left(H
      \left(\left(a^2+r^2\right)^2-a^2 {\mu }^2 \left(a^2+r (r-2
    M)\right)\right)+4 a M r\right)+a^2 \left({\mu }^2-1\right)+r (2
  M-r)}{r^2-a^2 \left({\mu }^2-1\right)},\\
  g_{\tau \tilde\varphi} &=&\frac{\mu ^2 \left(H
      \left(\left(a^2+r^2\right)^2-a^2 \mu ^2 \left(a^2+r (r-2
  M)\right)\right)+2 a M r\right)}{a^2 \left(\mu ^2-1\right)-r^2}.
\end{eqnarray}
In terms of ADM variables this only introduces a change in shift function
\begin{eqnarray}
  N_{\phi} = \frac{\mu ^{2}(2aMr+((a^{2}+r^2 )^{2}-a^{2}(a^{2}+(r-2M)
  )\mu ^{2}   )H) }{a^{2}(\mu^{2}-1 ) r^{2} }.
\end{eqnarray}
Using the new metric components in Horava's equations we will have
PDEs for the function $H(r,\mu )$ which are discussed in mathematica files.

\section{Hocava}

\begin{\eq}
  S_g &= & \int_{{\bf R} \times \Si_t} dt d^3 x
  \sqrt{g}N\left[\frac{1}{\kappa}\left(K_{ij}K^{ij}-\lambda
  K^2\right)+\xi R^{}-2 \La+\f{\sigma}{2} a_i a^i \right]\ ,
  \label{horava}
\end{\eq}
where
%\begin{\eq}
$
K_{ij}=({2N})^{-1}\left(\dot{g}_{ij}-\nabla_i
N_j-\nabla_jN_i\right)
$
% \end{\eq}
is the extrinsic curvature [the overdot $(\dot{})$ denotes the time
  derivative and $\nabla_i$ is the covariant derivatives for the
induced metric $g_{ij}$ on the time-slicing hypersurface $\Si_t$] and
$R$ is the {\it three}-curvature in the ADM metric
\begin{\eq}
  \label{metric}
  ds^2=-N^2 c^2 dt^2+g_{ij}\left(dx^i+N^i dt\right)\left(dx^j+N^j
  dt\right),
\end{\eq}
respectively. (Hereafter, we shall use the unit $c=1$, unless stated otherwise.)
The last term in the gravity action (\ref{horava}) is introduced for
completeness, with the proper acceleration $a_i=\nabla_i ln N$ \cite{Blas:2009}.

\section*{Acknowledgments}

Thanking part.

\appendix

\section{Explicit expressions of $\widehat{\rho}$ and $\widehat{p_a}$ in
  ${T^{\m}}_{\n (\it{eff})}$
  %the effective energy-momentum tensor components
}

%%%%%%%%%% References %%%%%%%%%%%%%%%%%%%%%%%%%
\newcommand{\J}[4]{#1 {\bf #2} #3 (#4)}
\newcommand{\andJ}[3]{{\bf #1} (#2) #3}
\newcommand{\AP}{Ann. Phys. (N.Y.)}
\newcommand{\MPL}{Mod. Phys. Lett.}
\newcommand{\NP}{Nucl. Phys.}
\newcommand{\PL}{Phys. Lett.}
\newcommand{\PR}{Phys. Rev. D}
\newcommand{\PRL}{Phys. Rev. Lett.}
\newcommand{\PTP}{Prog. Theor. Phys.}
\newcommand{\hep}[1]{ hep-th/{#1}}
\newcommand{\hepp}[1]{ hep-ph/{#1}}
\newcommand{\hepg}[1]{ gr-qc/{#1}}
\newcommand{\bi}{ \bibitem}
%%%%%%%%%%%%%%%%%%%%%%%%%%%%%%%%%%%%%%%%%%%%%%%

\begin{thebibliography}{999}

  \bibitem{Kerr:1963}
  R.~P.~Kerr,
  %``Gravitational field of a spinning mass as an example of
  % algebraically special metrics,''
  Phys.\ Rev.\ Lett.\  {\bf 11}, 237 (1963).
  % doi:10.1103/PhysRevLett.11.237

  %\cite{Chandrasekhar:1985kt}
  \bibitem{Chan:1985}
  S.~Chandrasekhar,
  ``The mathematical theory of black holes,'' (Oxford Univ.
    %ersity
  Press, 1983).
  %348 citations counted in INSPIRE as of 12 Jan 2024

  %\cite{Gibbons:2004uw}
  \bibitem{Gibb:2004}

  G.~W.~Gibbons \textit{et al.},
  %H.~Lu, D.~N.~Page and C.~N.~Pope,
  %``The General Kerr-de Sitter metrics in all dimensions,''
  J. Geom. Phys. \textbf{53}, 49
  %-73
  (2005), Appendix {\bf E}.
  %doi:10.1016/j.geomphys.2004.05.001
  %[arXiv:hep-th/0404008 [hep-th]].
  %424 citations counted in INSPIRE as of 12 Jan 2024

\end{thebibliography}

\end{document}
