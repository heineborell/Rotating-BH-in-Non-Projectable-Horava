%\documentclass[twocolumn,showpacs,preprintnumbers,amsmath,amssymb]{revtex4}
%\documentclass[preprint,aps,tightenlines,showpacs,nofootinbib]{revtex4}
\documentclass[preprint,aps,tightenlines,showkeys,nofootinbib,superscriptaddress]{revtex4}
\usepackage{latexsym}
\usepackage[dvips]{graphicx}
%  Mu-In's macros
%\def\rref#1{(\ref{#1})}
\newcommand{\beq}{\begin{eqnarray}}
\newcommand{\eeq}{\end{eqnarray}}
\newcommand{\bea}{\begin{eqnarray*}}
\newcommand{\eea}{\end{eqnarray*}}
\newcommand{\eq}{eqnarray}
\newcommand{\bb}{\bibitem}
\newcommand{\al}{{\alpha}}
\newcommand{\be}{{\beta}}
\newcommand{\ci}{\cite}
\newcommand{\ga}{{\gamma}}
\newcommand{\Ga}{{\Gamma}}
\newcommand{\ep}{{\epsilon}}
\newcommand{\epi}{{\epsilon^{ij}}}
\newcommand{\bepi}{\bar{\epsilon}^{ij}}
\newcommand{\bep}{{\bf \epsilon}}
\newcommand{\de}{{\delta}}
\newcommand{\De}{\Delta}
\newcommand{\ka}{\kappa}
\newcommand{\tD}{\tilde{\Delta}}
\newcommand{\Th}{{\Theta}}
\newcommand{\bT}{{\bf \Theta}}
\newcommand{\la}{{\lambda}}
\newcommand{\La}{{\Lambda}}
\newcommand{\m}{{\mu}}
\newcommand{\n}{{\nu}}
\newcommand{\si}{{\sigma}}
\newcommand{\Si}{{\Sigma}}
\newcommand{\om}{{\omega}}
\newcommand{\Om}{{\Omega}}
\newcommand{\pa}{{\partial}}
\newcommand{\no}{{\nonumber}}
\newcommand{\f}{\frac}
%\newcommand{\eep}{\epsilon_{b a_1 \cdots a_{n-1}}}
%\newcommand{\he}{{\hat{\epsilon}_{a_1 \cdots a_{n-1}}}}
%\newcommand{\hx}{{\hat{\xi}}}
%\newcommand{\bp}{\bar{\phi}}
%\newcommand{\wt}{\widetilde}
\newcommand{\ra}{\rightarrow}
\newcommand{\lra}{\leftrightarrow}
%\newcommand{\YM}{\hbox{\scriptsize YM}}
%\newcommand{\Higgs}{\hbox{\scriptsize Higgs}}
%\newcommand{\Dil}{\hbox{\scriptsize Dil}}
%\newcommand{\Lie}{\hbox{\scriptsize Lie}}
\newcommand{\Diff}{\hbox{\scriptsize Diff}}
%\newcommand{\Ramond}{\hbox{\scriptsize Ramond}}
%\newcommand{\NS}{\hbox{\scriptsize Neveau-Schwartz}}
\newcommand{\eff}{\hbox{\scriptsize eff}}
\newcommand{\tot}{\hbox{\scriptsize tot}}
\newcommand{\stat}{\hbox{\scriptsize stat}}
\newcommand{\exact}{\hbox{\scriptsize exact}}
\newcommand{\mmin}{\hbox{\scriptsize min}}
\newcommand{\new}{\hbox{\scriptsize new}}
\newcommand{\mmax}{\hbox{\scriptsize max}}
\newcommand{\AdS}{\hbox{\scriptsize AdS}}
\newcommand{\BTZ}{\hbox{\scriptsize BTZ}}
\newcommand{\HMTZ}{\hbox{\scriptsize HMTZ}}
\newcommand{\BH}{\hbox{\scriptsize BH}}
\newcommand{\Euc}{Euclidean }
\newcommand{\Sch}{Schwarzschild }
\newcommand{\ther}{thermodynamics }
\newcommand{\therl}{thermodynamical }
\newcommand{\mic}{micro-canonical }
\newcommand{\can}{canonical }
\newcommand{\lo}{logarithmic }
\newcommand{\temp}{temperature }
\newcommand{\cha}{characteristic}
\newcommand{\mln}{\mbox{ln}}
\newcommand{\hb}{\hat{\be}}
\newcommand{\hO}{\hat{\Om}}
\newcommand{\he}{\hat{\eta}}
\newcommand{\we}{\wedge}
\newcommand{\na}{\nabla}
\newcommand{\bn}{\bar{\nabla}}
\newcommand{\bg}{\bar{g}}
\newcommand{\fn}{\footnote}
\newcommand{\appr}{approximation}
\newcommand{\kI}{\kappa^{-1} \epi}
\newcommand{\GI}{(16 \pi G)^{-1} \epi}
\newcommand{\G}{\f{1}{16 \pi G} \epi}
\newcommand{\GG}{\f{2}{16 \pi G} \epi}
\newcommand{\mG}{\f{2 \m}{16 \pi G} \epi}
\newcommand{\dd}{\de^2 (x-y)}
\newcommand{\asy}{asymptotically}
\newcommand{\Asy}{Asymptotically}
\newcommand{\SchAdS}{Schwartzshild-AdS}
\newcommand{\SchdS}{Schwartzshild-dS}
\newcommand{\SchAAdS}{Schwartzshild-(A)dS}
\newcommand{\Ho}{Ho\v{r}ava}
\newcommand{\diff}{diffeomorphism}
\newcommand{\DiffF}{${\it Diff}_{\cal F}$}
%\newcommand{\sinh}{\mbox{sinh}}
%\newcommand{\sin}{\mbox{sin}}
%\newcommand{\cos}{\mbox{cos}}
%\newcommand{\cosh}{\mbox{cosh}}
%%%%%%%%%%%%%%% End of Private Macros %%%%%%%%%%%%%%%%%%%%%%%%%%

\begin{document}

\preprint{arXiv:2402.xxxx [hep-th]}

\title{Rotating Black Holes in
%Ho\v{r}ava Gravity:
%as
a Viable Lorentz-Violating Gravity:
 Finding
 %Kerr-type
 Exact Solutions Without Tears
%in Four-Dimensional Ho\v{r}ava Gravity%: The Massless Black Hole Case
}

\author{Deniz O. Devecio\u{g}lu \footnote{E-mail address: dodeve@gmail.com}}
%\affiliation{School of Physics, Huazhong University of Science and Technology,
%Wuhan, Hubei,  430074, China }

\author{Mu-In Park \footnote{E-mail address: muinpark@gmail.com, Corresponding author}}
\affiliation{ Center for Quantum Spacetime, Sogang University,
Seoul, 121-742, Korea }
\date{\today}

\begin{abstract}
We introduce a two-step procedure for finding Kerr-type rotating black hole solutions without tears.
Considering the low-energy sector of Ho\v{r}ava gravity as a viable Lorentz-violating gravity in four dimensions
which admits a different speed of gravity, we find the exact rotating black
hole solutions (with or without cosmological constant). We find that the
singular region extends to $r<0$ region from the ring singularity at $r=0$ in Boyer-Lindquist coordinates. There are two Killing horizons
where $g^{rr}=0$ and the black hole thermodynamics laws are still valid.
We find the rotating black hole solutions with electromagnetic charges
only when we consider the {\it noble} electromagnetic couplings, in
such a way that the speed of light is the same as the speed of gravity.
With the noble choice of couplings,
%we find that
our Lorentz-violating gravity can be consistent with the recently-observed time delay of the coincident GW and GRB signals. Furthermore, in the  Appendices, we show that (a) the solutions are the Petrov type I with four distinct principal null vectors and (b) the Hamilton-Jacobi equation for the geodesic particles are {\it not} separable.

\bigskip
\vspace{1cm}
\noindent
    \emph
   \it{Dedicated to the memory of Roman Jackiw  (8 November 1939 - 14 June 2023).
   %, the mentor.
   }

\end{abstract}

%\pacs{04.20.Jb, 04.20.Dw, 04.60.Kz, 04.60.-m, 04.70.Dy }
\keywords{Horava gravity, Rotating black hole solutions, Black hole thermodynamics, Gravitational waves, Lorentz violations}

\maketitle

\newpage

\section{Introduction}

The rotating black hole solution which was discovered by Kerr \cite{Kerr:1963} in general relativity (GR) has occupied an immovable position: The solution structure is quite rigid and it is extremely difficult to find another {\it exact} solution with non-trivial deformations by cracking the rigid solution structure. Moreover, even deriving the known Kerr solution is not quite easy task and requires some sophisticated methods whose generic validities in other gravity theories are not clear \cite{Chan:1985}.
%,Teuk:2014}. 
In particular, from the separability of Hamilton-Jacobi equation for the geodesic particles in a quite general context, Carter obtained the Carter constant as well as the Kerr solution \cite{Cart:1973}. Later, it was proved that the Petrov type-D solutions, like Kerr metric, guarantee the existence of a Killing tensor and its concrete form was derived in GR \cite{Walk:1970}

In this paper, we introduce a two-step procedure to find Kerr-type exact solutions without much tears. We apply the procedure
to the low-energy sector of Ho\v{r}ava gravity as a Lorentz-violating (LV)
gravity in four dimensions which admits a different speed
of gravity, and
%which has been proposed as a UV complete, renormalizable, %quantum
%gravity model without the ghost
%({\it i.e.,} unitarity)
%problem,
we find the exact rotating black hole solutions (with or without cosmological constant). In addition, if we consider charged rotating black holes, the {\it noble} electromagnetic couplings are needed for consistency, in such a way that the speed of light is the same as the speed of gravity. In Sec. II, we introduce our two-step procedure to find Kerr-type exact solutions. In Sec. III, we apply the procedure to the low-energy sector of Ho\v{r}ava gravity in four dimensions and find the exact rotating black hole solution without cosmological constant. In Sec. IV, we study the singularity structure which is richer than Kerr solution. In Sec. V, we study the horizon structure and new barriers for geodesic particles which are genuine to our LV gravity. In Sec. VI, we consider the generalization with cosmological constant and electromagnetic charges, and
%also
show that the first law of black hole thermodynamics is satisfied. In Sec. VII, we consider some observational constraints and in Sec. VIII, we conclude with some discussions.
In Appendix {\bf C}, we study the Petrov classification,
%the effects of Lorentz violations in the
the Killing tensor, and the Hamilton-Jacobi equation.

\section{A two-step procedure to find Kerr-type stationary solutions}

(i) Suppose that there is an {\it exact} ``massless" rotating space-time solution, which means it is the exact solution for an {\it arbitrary} rotation parameter $a$. This is the first step and the massless solution is a {\it seed} solution for the step 2.

(ii) The second step is to introduce the ``mass"-dependent ansatz functions into the {\it massless} seed solution,
without loss of much generality but still in a solvable way, {\it i.e.}, with
%without partial
ordinary differential equations for the polar angle $\theta$ or the radial coordinate $r$. This requires noble insights as well as some trials and errors. Without cosmological constant, we find that the situation becomes simpler since the
{\it massive} rotating solution does not deform the angle-dependent parts but just
%deform the space-time only in
the radial-dependent parts
%direction.
of the {\it massless} rotating space-time solution. On the other hand, with a cosmological constant, it is a bit more complicated and we need some more clever choice of the ansatz. So, in the following sections, we first consider the solution without cosmological constant and later with a cosmological constant.

The above two-step procedure can be applied to any gravity theory, in principle. In this paper, we apply the procedure to find the rotating black hole solutions in the low-energy sector of Ho\v{r}ava gravity in four dimensions, for the first time.

\section{Low-energy sector of Ho\v{r}ava gravity and rotating black hole solutions }

The low energy sector of (non-projectable) Ho\v{r}ava gravity \cite{DeWi:1967,Hora:2009} is described by the action, up to boundary terms,
\begin{\eq}
S_g &= & \int_{{\bf R} \times \Si_t} dt d^3 x
\sqrt{g}N\left[\frac{1}{\kappa}\left(K_{ij}K^{ij}-\lambda
K^2\right)+\xi R^{}-2 \La+\f{\sigma}{2} a_i a^i \right]\ ,
\label{horava}
\end{\eq}
where
%\begin{\eq}
$
 K_{ij}=({2N})^{-1}\left(\dot{g}_{ij}-\nabla_i
N_j-\nabla_jN_i\right)
$
% \end{\eq}
is the extrinsic curvature [the overdot $(\dot{})$ denotes the time derivative and $\nabla_i$ is the covariant derivatives for the induced metric $g_{ij}$ on the time-slicing hypersurface $\Si_t$] and $R$ is the {\it three}-curvature in the ADM metric
\begin{\eq}
\label{metric}
ds^2=-N^2 c^2 dt^2+g_{ij}\left(dx^i+N^i dt\right)\left(dx^j+N^j
dt\right),
\end{\eq}
respectively. (Hereafter, we shall use the unit $c=1$, unless stated otherwise.)
The last term in the gravity action (\ref{horava}) is introduced for completeness, with the proper acceleration $a_i=\nabla_i ln N$ \cite{Blas:2009}.

The equations from the variations of $N, N^i$, and $g^{ij}$ are given by
\beq
{\cal H}&\equiv&\f{1}{\kappa}\left(K_{ij}K^{ij} -\lambda K^2\right)-\xi R +2 \La
-\si \left(\f{1}{2} \f{\nabla_i N \nabla^i N }{N^2} -\f{\nabla_k \nabla^k N }{N} \right) =0\, , \label{eom1} \\
{\cal H}^i&\equiv&\f{2}{\ka}\nabla_j \left(K^{ji}-\lambda\,Kg^{ji}\right)=0\, ,\label{eom2}\\
E_{ij}&\equiv&\frac{1}{\kappa}\left( E_{ij}^{(1)}-\lambda E_{ij}^{(2)} \right)
+\xi E_{ij}^{(3)}
+\frac{\si}{2}E_{ij}^{(4)}=0, \label{eom3}
\eeq
where
\bea
E_{ij}^{(1)}&=& N_i \nabla_k K^k{}_j + N_j\nabla_k K^k{}_i -K^k{}_i
\nabla_j N_k-
   K^k{}_j\nabla_i N_k - N^k\nabla_k K_{ij}\no\\
&& - 2N K_{ik} K_j{}^k
  -\frac{1}{2} N K^{k\ell} K_{k\ell}\, g_{ij} + N K K_{ij} + \dot K_{ij}
\,,\no \\
E_{ij}^{(2)}&=& \frac{1}{2} NK^2 g_{ij}+ N_i \pa_j K+
N_j \pa_i K- N^k (\pa_k K)g_{ij}+  \dot K\, g_{ij}\,,\no\\
E_{ij}^{(3)}&=&N\Big(R_{ij}- \frac{1}{2}R g_{ij}+\frac{\La}{\xi}g_{ij}\Big)-(
\nabla_i\nabla_j-g_{ij}\nabla_k\nabla^k)N\,,\no\\
E_{ij}^{(4)}&=&\f{1}{N} \Big(-\f{1}{2} g_{ij} \nabla_k N \nabla^k N+ \nabla_i N \nabla_j N\Big)\,.
\eea
With the arbitrary parameters $\la, \xi$, and $\si$, the action does not admit the full diffeomorphism but only the {\it foliation-preserving} \diff~(${\it Diff}_{\cal F}$) \cite{Hora:2009,Park:2009},
\begin{\eq}
\delta_{\xi} t&=&-{\xi}^{0}(t),~~ \delta_{\xi} x^{i}=-\xi^{i}(t,{\bf x}), \no %\label{deltx_Horava}
\\
\delta_{\xi} N&=&(N{\xi}^{0})_{,0}+\xi^{k}\nabla_{k}N, \no %\label{delN3},
\\
\delta_{\xi}{N_{i}}&=&{\xi}^{0}{}_{,0}N_{i}+\xi^{j}{}_{,0}g_{ij}+\nabla_{i}\xi^{j}N_{j}
+N_{i,0}{\xi}^{0}+\nabla_{j}N_{i}\,\xi^{j}, \no %\label{delNi3},
\\
\delta_{\xi}{g_{ij}}&=&\nabla_{i}\xi^{k}g_{kj}+\nabla_{j}\xi^{k}g_{ki}
+g_{ij,0}{\xi}^{0}.
\label{delg3}
\end{\eq}
Note that the Einstein-aether theory reduces to the same gravity action (\ref{horava}) with the {\it hypersurface-orthogonal} aether field \cite{Jaco:2013} or the khronometric theory with a {\it global} time-like khronon field \cite{Blas:2011}. In this paper, we do not introduce additional assumptions on the aether or khronon field, but we only study the gravity action (\ref{horava}).

The first step toward the rotating black hole solutions is to find the {\it massless} rotating black hole solutions. Here, we first consider the
asymptotically {\it flat} case without cosmological constant and the $a_i$
extension, {\it i.e.}, $\La=0,~ \si=0$, for simplicity. To this end, we note
the recent finding by one of authors that the massless Kerr solution is also
a solution of Ho\v{r}ava gravity with an arbitrary $\la$ \cite{Park:2023},
%  with the spherical horizon topology $(k=+1)$
\beq
ds^2_{0}=-\frac{\rho^2 {\Delta}_r^{(0)} }{ {\Sigma}^2_{(0)}} dt^2+\frac{{\rho}^2}{ {\Delta}_r^{(0)}}dr^2+ {\rho}^2 d \theta^2
+\frac{{\Sigma}^2_{(0)} \mbox{sin}^2\theta}{{\rho}^2} d \phi ^2,
\label{massless_Kerr}
\eeq
where,
%\begin{eqnarray}
$
{\rho}^2 = r^2 + a^2 \mbox{cos}^2 \theta,
%\nonumber \\
{\Delta}_r^{(0)} = \left( r^2 + a^2 \right) ,
%\no \\
{\Sigma}^2_{(0)} = \left( r^2 + a^2 \right) {\rho}^2,
$
%\end{eqnarray}
and $(t,r,\theta, \phi)$ are the Boyer-Lindquist coordinates
% \cite{Boye:1967}
\ci{Cart:1973,Gibb:1977,Park:2001}. The metric (\ref{massless_Kerr})
is nothing but the {\it flat} Minkowski spacetime, but written in the {\it ellipsoidal} spacial
coordinates, which are related to Cartesian coordinates
$x=(r^2+a^2)^{1/2} \mbox{sin}\theta~ \mbox{cos} \phi,
y=(r^2+a^2)^{1/2} \mbox{sin} \theta~ \mbox{sin} \phi,
z=r~\mbox{cos} \theta$.
% (Fig. 1)

%\begin{figure}
%\includegraphics[width=5cm,keepaspectratio]{ellipsoid.eps}
%\qquad
%\includegraphics[width=7cm,keepaspectratio]{.eps}
%\caption{Ellipsoidal coordinates for a fixed $\phi$ coordinate, used in Kerr metric (\ref{massless_Kerr}). Here, $r=0$ is a two-dimensional disk with a circle boundary at $\theta=\pi/2$. There are two copies of the coordinate systems for both $r>0$ and $r<0$, with the connecting boundary at $r=0$. }\label{fig:ellipsoidal}
%\end{figure}

Here, it is important to note that $r=0$
is not the end of the coordinates but there is another copy of Kerr spacetime
in the $r<0$ regime, with another asymptotic infinity at $r\ra -\infty$ in ellipsoidal coordinates system. Actually, the massless Kerr metric (\ref{massless_Kerr}) was interpreted as a wormhole solution in \cite{Gibb:2017}. Of course, there is no event horizon in (\ref{massless_Kerr}) but {\it the genuine spacetime deformation for a rotating object is believed to be naturally encoded in the ellipsoidal coordinates with the rotation parameter $a$.}

The next step towards the generic rotating black hole solutions is to consider the mass-dependent ansatz functions without loss of generality, {\it until we may obtain the consistent solutions}. In the known Kerr metric, the mass term is tightly bounded to the rotation parameter $a$ and it is not easy to separate them. So, it is important to consider a generic ansatz such that the spinning part and the mass part are consistently separated which might crack the rigid structure of Kerr solution. Now, by comparing to Kerr solution, we consider the metric ansatz,
\beq
ds^2_{1}=-N^2 dt^2+\frac{\rho^2}{ \Delta_r}dr^2+ \rho^2 d \theta^2
+\frac{ \Sigma^2 \mbox{sin}^2\theta}{\rho^2} \left(d \phi +N^{\phi} dt \right)^2,
\label{massive_Kerr_metric}
\eeq
where,
\beq
\Sigma^2&=& \left( r^2 + a^2 \right) \rho^2 +f(r) a^2 \mbox{sin}^2\theta, \no \\
N^2&=&\frac{\rho^2 \Delta_r (r) }{ \Sigma^2},~~ %\no \\
N^{\phi}=-\f{g(r)}{\Sigma^2}
\label{massive_Kerr_functions}
\eeq
with three $r$-dependent undetermined functions $f(r), g(r)$, and $\Delta_r (r)$. Then, by solving the full equations of motion (\ref{eom1}-\ref{eom3}) for the ansatz (\ref{massive_Kerr_metric}), we can determine the undetermined functions uniquely as,
\beq
f(r)=2m r, ~g(r)=2 {\it a} mr \sqrt{\kappa \xi}, ~\Delta_r (r)=r^2+a^2-2mr,
\label{f,g_sol}
\eeq
which reduces to the known Kerr solution in the GR case of $\xi=1/\ka$ or Schwarzschild solution with $a=0$, where $m$ is an integration constant parameter. Here, we adopt the usual choice of $N^{\phi}|_{\infty}=0, W(\infty) \equiv N \sqrt{g_{rr}}|_{\infty}=1$. It is rather surprising that the Kerr-solution cracking term with the LV factor $\sqrt{\kappa \xi}$ appears only in $N^{\phi}$. But, if we look at the component form
\beq
ds^2_{1}&=&\left[ -\f{(\De_r-a^2 \mbox{sin}^2\theta )}{\rho^2}+\f{(\ka \xi-1)~(2mr)^2~ a^2 \mbox{sin}^2\theta }{\rho^2 \Si^2}\right] dt^2 \no \\
&+&\frac{\rho^2}{ \Delta_r}dr^2+ \rho^2 d \theta^2
+\frac{ \Sigma^2 \mbox{sin}^2\theta}{\rho^2} d \phi ^2-\f{4 {\it a} mr \sqrt{\ka \xi}~ \mbox{sin}^2\theta}{ \rho^2} dt d\phi,
\label{massive_Kerr_metric_comp}
\eeq
one can easily see the non-trivial LV
%(LV)
effect for $\xi \neq 1/\ka$
in $g_{tt}$ as well as in $g_{t \phi}$ components. Another notable property is that our
solution, as well as the Kerr solution with $\xi = 1/\ka$, are valid for
an {\it arbitrary} $\la$ due to $K=0$, {\it i.e.,} ``maximal"
%time-
slicing. This makes even Kerr solution or Schwarzschild solution with $a=0$ has different notions of singularities as we can see in the next section, due to the lack of the full {\it Diff} with $\la \neq 1$.

The mass and angular momentum are computed as
\beq
M=16 \pi \xi m, ~ J=16 \pi {\it a} m \sqrt{ \xi \ka^{-1}},
\eeq
by the standard definition \cite{Bana:1992}, as the conjugates to $W(\infty)$ and $N^{\phi} (\infty)$, respectively, in the boundary term
\beq
{\cal B}=(t_2-t_1) [-W(\infty) M+N^{\phi} (\infty) J ]
\eeq
which makes the total action $S=S_g +{\cal B}$ to be {\it differentiable} with the boundary
conditions $\de N^{\phi} (\infty)=\de W(\infty)=0$. We obtain the same result in the
recently-proposed conserved-charge formula in a {\it covariant} formalism
\cite{Deve:2021}
%\footnote{There were some typos in the charge formula of \cite{Deve:2021}.}
and it provides another non-trivial evidence of the formalism. The generalizations
to the asymptotically {\it (A)dS} black holes ($\La \neq 0$) and the {\it charged} black holes will be
considered in the later part, due to some complications. However, with the $a_i$
extension term, we find
%that there is
no solution for the ansatz (\ref{massive_Kerr_metric}), except the massless solution, which indicates that, with the $a_i$ term, the solution structure will be quite different than ours or GR. So, from now on we consider only the standard case without $a_i$, {\it i.e.,} $\si=0$.

\section{Singularity structure}

Due to the {\it apparent} lack of the full {\it Diff}, we have a different notion of singularities. In our case, the physical singularities are captured by \DiffF~ invariant
curvatures 
%(note 
($K=0,~K_{ij} K^{R}=0$),
\beq
R \sim \f{a^2 m^2}{\rho^6 \Si^4},~R_{ij} R^{ij} \sim \f{m^2}{\rho^{12} \Si^8},~K_{ij} K^{ij} \sim \f{\ka \xi a^2 m^2}{\rho^6 \Si^4},
\label{3D_curvature}
\eeq
where we have omitted the other detailed factors which are finite and
are not canceled by the denominators. It is interesting to note that the
omitted factors are not modified by our solutions and they are exactly the
same as in GR. The distinct feature in our case is that the singularity
properties of each quantity in (\ref{3D_curvature}) are physically meaningful
and they show the curvature singularities at $\Si^2=0$, as well as the usual
%(ring)
singularity at $\rho^2=0$. On the other hand, the four-dimensional curvature invariants in GR are given by
\beq
&&R^{(4)} \sim (\ka \xi-1)~ \f{a^2 m^2}{\rho^6 \Si^4},~R^{(4)}_{\m \n} R^{(4) \m \n} \sim (\ka \xi-1)^2~ \f{a^4 m^4}{\rho^{12} \Si^8},\no \\
&&R^{(4)}_{\m \n \si \rho} R^{(4) \m \n \si \rho} \sim (\ka \xi-1)~ \f{m^2}{\rho^{12} \Si^8}+\f{m^2}{\rho^{12}} \left(\cdots\right),
\label{4D_curvature}
\eeq
where the omitted term in $\left(\cdots\right)$ is the same as in GR. This clearly shows that the additional singularities of (\ref{3D_curvature}) at $\Si^2=0$ are exactly canceled in the GR limit $\xi =\ka^{-1}$ so that only the usual
ring singularity at $\rho=0$, {\it i.e.,} $r=0, \theta=\pi/2$ remains in GR \footnote{This is a common feature in \Ho~gravity \cite{Lu:2009,Park:2012}, as was once noted by S. Mukohyama earlier (a private communication). }. Furthermore, even in the GR limit $\xi =\ka^{-1}$, {\it i.e.}, Kerr solution or even \Sch solution, due to the physical meaningfulness of \DiffF~ invariant quantities in (\ref{3D_curvature}), not (\ref{4D_curvature}), we still have the same additional singularities at $\Si^2=0$.

Fig. 1 shows the singularity surfaces of
\beq
\mbox{cos}^2\theta=\f{2mr a^2 +r^2 (r^2+a^2)}{a^2 (2 mr -(r^2+a^2))}
\eeq
for $\Si^2=0$, by varying the mass parameter $m$ and the rotation parameter $a$. The lower $(r<0)$ and the upper $(r>0)$ branches correspond to $m>0$ and $m<0$, respectively (the left panel).

As $m \ra \pm \infty$, the singularity surfaces cover the whole region of $r<0$ or $r>0$, with $r_0=\mp a/\sqrt{3}$ which touches the $\theta$ boundaries $\theta=0, \pi$
%, and with the maximum of $r$, $r_{max}=\mp \infty$ at $\theta=\pi$
(the right panel).
As $a \ra 0$, the singularity surfaces reduce to the point singularity of \Sch black hole at $r=0$. Other than these extreme cases, {\it i.e.,} $0<a \leq m <0$, there always exits a time-like trajectory which avoids the singularity surfaces such that the closed-time-like curves (CTC) are possible, similarly to Kerr solution in GR.

\begin{figure}
\includegraphics[width=7cm,keepaspectratio]{singularity.eps}
\qquad
\includegraphics[width=7cm,keepaspectratio]{r0.eps}
\caption{Left: $r$ vs. $\theta~ [0, \pi]$ of singularity surfaces for $\Si^2 (r, \theta)=0$, by varying $a=2,1,0.1$ (from the outer to inner curves) with $m=2$ (the lower branch), $m=-2$ (the upper branch). Right: The extreme radius $r_0$ vs. $m$, where $\theta$ becomes an extreme value for a given $a$ and $m$, and
$r_0
%approaches to
\ra \mp a/\sqrt{3}$ as $m \ra \pm \infty$.}\label{fig:singular}
\end{figure}

\section{Killing horizons and particle barriers}

In our coordinates, there are two event horizons at $r_{\pm}$ where $g^{rr}=\Delta_r/\rho^2=0$, which occurs when
\beq
\De_r =r^2+a^2-2ma=0,
\eeq
since $\rho^2 >0$, excluding the singularity at $\rho^2=0$ that was studied in Sec. IV. The event horizons at $r_{\pm}$ are Killing horizons with the Killing vectors $\chi^a_{\pm}=(1,0,0,\Om_{\pm})$, which become {\it null} vectors $\chi^a_{\pm} {\chi_a}_{\pm}=-\rho^2 \De_r/\Si^2+\Si^2 \mbox{sin}^2\theta (N^{\phi}-\Om_{\pm})^2/\rho^2=0$ for the {\it constant} angular velocities on the horizons $r_{\pm}$, $\Om_{\pm}=-N^{\phi}|_{r_{\pm}}$. Then, one can define the surface gravities $\ka_{\pm}$ such that $\chi^{\nu}_{\pm} \nabla_{\nu}^{(4)} \chi^{\mu}_{\pm}=\ka_{\pm} \chi^{\mu}_{\pm}$ {\it on} the horizons $r_{\pm}$ \footnote{The direct computation in our coordinates is rather tricky and we first need to consider $\chi^{\nu}_{\pm} \nabla_{\nu}^{(4)} \widehat{\chi}^{\mu}_{\pm}=\ka_{\pm} \widehat{\chi}^{\mu}_{\pm}$ with $\widehat{\chi}^{\mu}_{\pm}=(1,\f{\De_r}{r^2+a^2},0,\Om_{\pm})$ and then consider the limit $\widehat{\chi}^{\mu}_{\pm} \ra \chi^{\mu}_{\pm}$ at the horizons.}.

From the {\it hypersurface-orthogonality} of the Killing horizons, one can find that $\ka_{\pm}$ are constants on the corresponding horizons $r_{\pm}$, {\it i.e.},
\beq
\chi^{[ \m} \nabla^{\n ]}_{(4)} \ka_{\pm}=-\chi^{[ \m} {R^{\n ]}}_{ \si (4)} \chi^{\si}= -\chi^{[ \m} {T^{\n ]}}_{\si ({\it eff}) } \chi^{\si}=0,
\label{0th law}
\eeq
where we have expressed our non-covariant equations (\ref{eom1}-\ref{eom3}) into
the {\it covariant} from
$R_{\m \n}^{(4)} -(1/2) g_{\m \n}^{(4)} R^{(4)}=T_{\m \n}^{({\it eff})}$ with
an {\it effective} energy-momentum tensor,
\beq
{{T}^{\m}}_{\n (\it{eff})}=\left( \begin{array}{cccc}
\widehat{\rho} & 0 & 0 & 0 \\
0 & \widehat{p_1} & \widehat{p_2} &0 \\
0 & \widehat{p_3} & -\widehat{p_1} & 0 \\
\widehat{p_4} & 0 & 0 & -3 \widehat{\rho},
\label{T_eff}
\end{array}\right)
\eeq
which is a {\it non-perfect} fluid form
%(we omitted the explicit expressions of $\widehat{\rho}$ and $p_i$, which are rather long and complicated)
(see Appendix {\bf A} for the explicit expressions of $\widehat{\rho}$ and $\widehat{p_a}$),
but still satisfies the covariant conservation equation $\nabla_{\m}^{(4)} T^{\m \n}_{({\it eff})}=0$.
The effective energy-momentum violates the dominant energy condition,
${T^{0}}_{ 0 (\it{eff})}\geq | {T^{\m}}_{\n (\it{eff})}|$, especially by $|{T^{3}}_{3 (\it{eff})}|=3 |\widehat{\rho}|>\widehat{\rho}$, but it satisfies $\chi^{[ \m}{T^{\n ]}}_{\si (\it{eff})} \chi^{\si}=0$
%(\widehat{\rho}+\widehat{p_4})~ (g_{tt} +\Om_{\pm} g_{t \phi})-3 \widehat{\rho}~ ( g_{t \phi}\Om_{\pm}+ g_{\phi \phi})=0$ 
on the horizons $r_{\pm}$ so that the {\it zeroth} law (\ref{0th law}) is satisfied. In other words, the usual notion of Killing horizons at $r_{\pm}$ satisfying the zeroth law are still valid in our LV gravity, which seems to be a quite non-trivial result since the effective energy-momentum tensor $T_{\m \n}^{(\it{eff})}$ might break the last equation of (\ref{0th law}), generally.

Now, we turn to the discussion on the role of the event horizons to the geodesic particles to see whether they have the similar roles even in LV gravity also. To this end, we consider the conserved energy $E=-v^{\m} (\partial_t)_\m$ and angular momentum $L=v^{\m} (\partial_{\phi})_\m$, for the time-like Killing vector $(\partial_t)_\m$ and the axial Killing vector $(\partial_{\phi})_\m$, and the {\it on-shell} equations
\beq
-\ep =g_{\m \n} v^{\m} v^{\n},
\label{geodesic}
\eeq
where $v^{\m} ={x^{\m}}'
%\dot{x}^{\m}
\equiv d x^{\m}/ d \tau$ for a particle's $4$-velocity with the proper time $\tau$, and $\ep=+1~ (-1)$ for a time-like (space-like) geodesics and $\ep=0$ for a null geodesics.

Here, it is important to note that {\it the line element (\ref{metric}) describes the invariant distance measure even in our Lorentz violating case}, contrary to a widespread belief based on the symmetry transformation of action which allows several LV terms when we consider the {\it reduced} \diff~\DiffF. Actually, one can prove that the {\it invariant} line element is only given by  (\ref{metric}), due to mixing between different components under the coordinate transformation, even if we consider \DiffF~(see Appendix {\bf B} for the proof). This justifies $E$ and $L$ as the conserved quantities, {\it independently on the chosen coordinates}, even in \Ho~ gravity where the full {\it Diff} broken into \DiffF, which has {\it never} been {\it known} in the literature.

Then, by solving ${t'}, {\phi'}$, and ${r'}$ in terms of the conserved quantities $E$ and $L$, one obtains
\beq
{t'}&=&\f{\left[(r^2+a^2) \rho^2+2mra^2 \mbox{sin}^2 \theta\right] E-2mra \sqrt{\ka \xi} L}{\rho^2 \De_r},\\
{\phi'}&=& \f{2mra \sqrt{\ka \xi} E+g_{tt} \rho^2 L/\mbox{sin}^2 \theta}{\rho^2 \De_r},\\
{r'}&=&\pm \sqrt{-2 \left(V+g_{\theta \theta} {\theta'}^2/2 g_{rr} \right)},
\eeq
where
\beq
V&=&-\f{\ep mr}{\rho^2}+\f{L^2}{2 \mbox{sin}^2\theta \rho^2}+(\ep-E^2) \f{(r^2+a^2)}{2 \rho^2}-\f{mr}{\rho^4} \left(-a E \mbox{sin}\theta+\f{L}{\mbox{sin}\theta} \right)^2 \no \\
&&+\f{ 2 (\sqrt{\ka \xi}-1) mra  E L}{\rho^4}- \f{(\ka \xi-1)  (2 mra)^2 L^2}{2 \rho^4 \Si^2}.
\eeq
For a complete analytic integration of the geodesics, including $\theta$ direction, we would need another constant of motion (we will discuss more about this in Appendix {\bf C}) but there is an important case where we can neglect ${\theta'}$, that is the {\it equatorial} geodesics, $\theta=\pi/2$: For other angles, $\theta \neq \pi/2$, the {\it full} geodesic equation (\ref{geodesic}) tells us that the $\theta=constant$ plane is not maintained generally, due to ${\theta''} \neq 0$ even if ${\theta'}=0$ initially.

Then, for the equatorial and the {\it co-rotating} geodesics with $L=a E$, which corresponds to the radial geodesics in non-rotating geometries \cite{Chan:1985}, we obtain [$+~ (-)$ denotes the out (in)-going geodesics]
\beq
\f{dt}{dr}&=&\pm \f{(r^2+a^2)-2 (\sqrt{\ka \xi}-1) ma^2 r^{-1} }{\De_r \sqrt{-U}},\label{dtdr}\\
\f{d \phi}{ dr} &=& \pm \f{a \left[ 1+2 (\sqrt{\ka \xi}-1) m r^{-1} - (\ka \xi-1) (2ma)^2 \left( (r^2+a^2)r^2+2 mr a^2\right)^{-1}\right]}{\De_r \sqrt{-U}},\label{dpdr}
\eeq
where
\beq
-U=-\f{\ep \De_r}{E^2 r^2}+1-\f{ 4 (\sqrt{\ka \xi}-1) ma^2 }{r^3}
+\f{(\ka \xi-1)  (2 ma^2)^2}{r^2 [(r^2+a^2) r^2 +2mr a^2]}.
\eeq
Note that (\ref{dtdr}) and (\ref{dpdr}) show the characteristic behaviors of the {\it light-cones}, which ``close up" or ``peel off", as the event horizons at $r_{\pm}$ are approached with $\De_r\ra 0$, especially for the null geodesics $(\ep=0)$, {\it independently} on particle's energy $E=L a^{-1}$.

\begin{figure}
\includegraphics[width=7cm,keepaspectratio]{barrier_null-like.eps}
\qquad
\includegraphics[width=7cm,keepaspectratio]{barrier_time-like.eps}
\caption{Mass parameter $m$ vs. horizons $r_{\pm}$ (the top curves) or barriers (the bottom curves) for null (the left panel) and time-like (the right panel) trajectories. The bottom curves denote the barrier regions by varying $E$ and $\ka\xi$: For the null trajectories (left panel), we take $\ka\xi=1.5,4,10$ (left to right); for the time-like trajectories (right panel), we take $\ka\xi=2, E=0.3$, and $\ka\xi=2,4,10, E=2$ (left to right). For time-like cases (right panel), the orbits are bounded for $E<\ep$ (yellow curve). The barriers exist even when the event horizons $r_{\pm}$ are absent when $m<m_*$, with the extremal black hole mass $m_*$, where two horizons merge.}
\label{fig:barrier}
\end{figure}

Moreover, with the coupling $\ka \xi > 1$ for null particles but arbitrary $\ka \xi$ for time-like particles, it
%also
shows a potential barrier, {\it i.e.}, $U > 0$. From the condition that the barrier should not exist outside the even horizon $r_+$ so that the identity of black holes can be still maintained, one finds the condition $\ka \xi \leq 4$, which gives its physical upper bound (Fig. 2): In the GR limit of $\ka \xi = 1$, the barrier also exists for time-like cases (the right panel) though absent in null cases (the left panel), but it exists always inside the event horizons and so it is not harmful. It is notable also that, for ingoing geodesics, the barrier can act as a {\it classical} protector of the curvature singularities at $r \leq 0$ with $m>0$. On the other hand,
for outgoing geodesics, it can produce thermal radiations via {\it quantum-mechanical} tunneling from the inner region even when the event horizon is absent.
%though quantum gravity and back-reaction need to be considered also in that regime,

\section{Generalizations with cosmological constant and electromagnetic charges}

Generalizing our rotating solution to the \asy~ {\it(A)dS} spacetime with a cosmological constant $\La$ is not quite straightforward and needs a quite clever choice of the reference solution, as given by \cite{Gibb:2004} \footnote{We thank O. Sarioglu for informing \cite{Gibb:2004}, which turns out to be a good reference solution for {\it Kerr-(A)dS} solutions.}. Starting from the Kerr-(A)dS solution of \cite{Gibb:2004}, and doing the same two-step procedure as in the previous sections, with the first step of finding the massless solution
%has been also
done also in \cite{Park:2023}, we obtain the rotating {\it (A)dS} black hole solutions as
\beq
ds^2=-N^2 dt^2+\frac{\rho^2}{\Delta_r (r)}dr^2+ \frac{\rho^2}{\Delta_{\theta} (\theta)} d \theta^2
+\frac{\Sigma^2 \mbox{sin}^2\theta}{\rho^2 \Xi^2 }
\left(d\phi+ N^{\phi} dt \right)^2,
\label{KAdS_ansatz}
\eeq
where
\begin{eqnarray}
{\Xi}&=&1+\f{\La a^2 }{ 3 \xi}, ~
% \nonumber \\
{\Delta}_\theta = 1+\f{\La a^2 \mbox{cos}^2 \theta}{ 3 \xi},\nonumber \\
{\Delta}_r &=& \left( r^2 + a^2 \right) \left(1 -\f{\La r^2}{3 \xi}\right)-2mr, \no \\
{\Sigma}^2 &=& \left( r^2 + a^2 \right) \rho^2 {\Xi}+f(r) a^2 \mbox{sin}^2\theta, \no \\
%\left( r^2 + a^2 \right)^2 \De_\theta - a^2 \mbox{sinh}^2 \theta \Delta_r,
N^2 &=&\frac{\rho^2 \Delta_r \De_\theta}{\Sigma^2},~
%\no \\
N^{\phi}=-\f{g(r) \De_{\theta} }{\Si^2}
\label{KAdS_sol}
\end{eqnarray}
with the same solutions for $f(r)$ and $g(r)$ as in (\ref{f,g_sol}).

The mass and angular momentum can be computed  as
\beq
M=\f{16 \pi \xi m}{\Xi^2}, ~J=\f{16 \pi a m \sqrt{\xi \ka^{-1}}}{\Xi^2},
\label{M,J_KdS}
\eeq
similarly to the $\La=0$ case in Sec. III, and they satisfy the first law of black hole
thermodynamics,
\beq
dM=T_H dS_H +\Om_{H} dJ
\label{1st law}
\eeq
with the horizon entropy,
\beq
S_H=\f{\xi (r_{H}^2+a^2)}{4 \hbar}
\eeq
and the Gibbons-Hawking temperature $T_H$ at the black hole horizons $r_{\pm}$ and the cosmological horizon $r_{++}$
%\cite{Gibb:1977}
\footnote{Here, we take the convention of the {\it negative} temperatures for
the inner and cosmological horizons
%, for convenience
\cite{Curi:1979,Klemm:2004,Park:2006,Cveti:2018}.}, and angular velocity at the horizons $\Om_H$,
\beq
T_H&=&\f{\ka_{H} \hbar}{2 \pi}=\f{r_H \left(1-\f{a \La}{3 \xi}-\f{r_H^2 \La}{\xi}-\f{a^2}{r_H^2} \right)}{4 \pi(r_H^2+a^2)}, \no \\
\Om_{H}&=&-N^{\phi}|_H=\f{a \left(1-\f{r_H^2 \La}{3 \xi} \right) \sqrt{\ka \xi}}{r_H^2+a^2}.
\eeq

Note that $\Om_{\infty}=-N^{\phi}|_{\infty}=0$, contrary to another standard form of Kerr-(A)dS solution in GR \cite{Cart:1973,Gibb:1977,Hawk:1998,Henn:1985}, and we obtain the correct $\Om_H$ for the first law of thermodynamics, without the background subtraction $\Om_H-\Om_{\infty}$. We obtain the correct mass and angular momentum from the standard definition or from the recently-proposed charge formula for a covariant formalism, consistently with the first law of thermodynamics \cite{Cald:1999,Gibb:2004b}. Moreover, due to the {\it r-independence} of the conserved charge
%formula
in \cite{Deve:2021}, we obtain the mass and angular momentum without the conceptual problem of the infinite boundary in the standard method for the rotating {\it dS} solutions.

Another standard form of rotating (A)dS black hole solutions with a non-vanishing angular velocity $\Om_{\infty}$ at infinity \cite{Cart:1973,Gibb:1977,Henn:1985,Hawk:1998,Cald:1999} is obtained by considering
\beq
\widehat{N}^{\phi}&=&-\f{g(r) \De_{\theta} }{\Si^2}-\f{a \La}{3 \xi},
\eeq
via a coordinate transformation $d \phi=d \widehat{\phi}-(a \La/3 \xi)dt$ in
the metric (\ref{KAdS_ansatz}), which
corresponds to a \DiffF \cite{Cart:1973,Henn:1985,Park:2023}.
%and (\ref{KAdS_sol}).
We find the same mass and angular momentum as (\ref{M,J_KdS}) and the first law of black hole thermodynamics as (\ref{1st law}), but now with the subtracted angular speed $\widehat{\Om}_{H}=-\widehat{N}^{\phi}|_H+\widehat{N}^{\phi}|_{\infty}$, in agreement with GR's result \cite{Hawk:1998,Henn:1985}.

Adding the electromagnetic charges to the rotating black hole solutions by coupling Maxwell fields to \Ho~ gravity is also an important generalization. To this ends, we consider the LV Maxwell action, up to boundary terms,
\beq
S_M=\int_{{\bf R} \times \Si_t} dt d^3 x \sqrt{g} N \left[ -\f{2 \eta}{N^2} \left(E_i +F_{ij}N^j \right)^2+\zeta F_{ij} F^{ij}\right],
\eeq
which admits \DiffF,~
%\beq
$
\de_{\xi} A_0 =(A_0 \xi^0)_{,0}+\xi^k \nabla_k A_0+A_k \xi^k_{,0},
%\no \\
~\de_{\xi} A_i =A_{i,0}\xi^0+\xi^k \nabla_k A_i+A_k \nabla_i\xi^k,
%\label{Diff_A}
$
%\eeq
%the same
as well as (\ref{delg3}) for the \Ho~gravity action (\ref{horava}),
where $E_i=\dot{A}_i -\nabla_i A_0, F_{ij}=\nabla_i A_j-\nabla_j A_i$,
and $\eta, \zeta$ are the electromagnetic coupling constants
\cite{Suda:1992,Kiri:2009,Lin:2014}.
 %and the Lorentz-violating parameter, respectively.
Then, solving the full equations from variations of $A_0$ and $A_i$, as well as $N, N^i$, and $g^{ij}$ as in Sec. III (see Appendix {\bf D} for the details), we obtain the gauge potential, with electric charge $q_e$ and magnetic charge $q_m$,
\beq
A=-\sqrt{\f{\xi}{\eta}} \f{\left[ q_e r \De_\theta + q_m a ~\mbox{cos}\theta~ (1-\La r^2 (3 \xi)^{-1})\right]}{\rho^2 \Xi} dt+ \f{1}{\sqrt{\eta \ka}} \f{\left[q_e r {\it a}~ \mbox{sin}^2\theta+q_m (r^2+a^2)~ \mbox{cos}\theta \right]}{\rho^2 \Xi} d \phi
\label{A_sol}
\no \\
\eeq
and the metric (\ref{KAdS_ansatz}), (\ref{KAdS_sol} by replacing $2mr\ra2mr-(q_e^2+q_m^2)$, but {\it only with} the ``noble"
%electromagnetic
coupling,
\beq
\zeta \eta^{-1}=\ka \xi.
\label{noble_coupling}
\eeq
It is rather surprising that the exact charged black hole solutions exist only in the noble coupling (\ref{noble_coupling}), due to coupling to \Ho~ gravity \footnote{It is interesting that the noble coupling can be also obtained from a Kaluza-Klein reduction from $(4+1)$-dimensional {\it kinetic-conformal} pure \Ho~ gravity $(\la=1/4)$ \cite{Rest:2019}.}. Actually, this is the case where the speed of gravity is identical to the speed of light (or electromagnetism), $c_g=c_l=\sqrt{\ka \xi}$ with the dispersion relation $\om^2= (\ka \xi) \bf{k}^2$ for the perturbations around the flat Minkowski background \cite{Park:2009}.

The first law of thermodynamics (\ref{1st law}) is extended as (cf. \cite{Cald:1999}), with the same $S_H$ and $\Om_H$,
\beq
dM=T_H dS_H +\Om_{H} dJ+\Phi^{(e)}_H dQ_e+\Phi^{(m)}_H dQ_m
\label{1st law_charged}
\eeq
with the Hawking temperature $T_H$, the electric and magnetic chemical potentials $\Phi^{(e,m)}_H$,
\beq
T_H&=&\f{r_H \left(1-\f{a \La}{3 \xi}-\f{r_H^2 \La}{\xi}-\f{a^2+q_e^2+q_m^2}{r_H^2} \right)}{4 \pi(r_H^2+a^2)}, \no \\
\Phi^{(e)}_H&=&16 \pi \sqrt{\xi \eta}~\f{q_e r_H}{r_H^2+a^2}, ~
\Phi^{(m)}_H=16 \pi \xi \sqrt{\ka \eta}~\f{q_m r_H}{r_H^2+a^2},
\eeq
and the canonical charges, computed from fluxes of electromagnetic fields,
\beq
Q_e=\sqrt{\f{\xi}{\eta}}\f{q_e}{\Xi}, ~Q_m=\f{1}{\sqrt{\ka \eta}}\f{q_m}{\Xi},
\eeq
which are also {\it r-independent} \footnote{For {\it dyonic} black holes, electric (magnetic) canonical charge has ``$r-$dependent" and ``induced" magnetic (electric) charge contribution also due to rotation, but they are exactly canceled and have no effects in (\ref{1st law_charged}).}, as in $M$ and $J$ (\ref{M,J_KdS}) in gravity.
On the other hand, another standard form of rotating (A)dS black hole solutions
with a non-vanishing angular velocity $\Om_{\infty}$ at infinity
%\cite{Cart:1973,Gibb:1977,Klem:1998,Hawk:1998,Henn:1985}
is obtained by
%\beq
$\widehat{A}=A-({a \La}/{3 \xi}) A_{\phi}dt$,
%\eeq
considering a coordinate transformation $d \phi=d \widehat{\phi}-(a \La/3 \xi)dt$ also.

The ergo-sphere is larger $(\ka \xi >1)$ or smaller $(\ka \xi <1)$ than that of GR. In particular, for the \asy~flat case, the ergo-surface ($g_{tt}=0$) is given by
\beq
\mbox{sin}^2\theta=\f{
\left\{ (r^2+a^2)^2+\De_r^2+(\ka \xi-1) (2mr)^2
-\sqrt{\left[ \left(r^2+a^2+\De_r \right)^2+ (\ka \xi-1) (2mr)^2 \right] \ka \xi (2mr)^2} \right\}
}{2 a^2 \De_r},\no \\
\eeq
which exists {\it always} for $\ka \xi>0$, from $g_{tt}|_{r_+}=(\ka \xi) a^2 \mbox{sin}^2 \theta /\rho^{2}|_{r_+}>0$. If we assume the {\it area non-decreasing} case with the appropriate energy conditions \cite{Hawk:1971}, one can consider the Penrose process \cite{Penr:1969} to extract positive energy from inside of the ergo-surface ($g_{tt}< 0$) and obtain {\it formally} the same energy-radiating efficiency as in Kerr metric, due to the same area formula $A=8 \pi m (m+\sqrt{m-a^2})$. However, its wave analogue, known as the {\it super-radiant} scattering is not quite clear, due to the lack of separability of wave equations (see Appendix {\bf C}).

\section{Observational constraints}

The recent observation of the arrival delay of $(+1.74 \pm 0.05)s$ in the  {\it coincident} gravitational waves (GW) and gamma rays (GW170817, GRB170817A) \cite{LIGO:2017} yields the constraints on the difference of the gravity speed $c_{g}$ and the gamma-ray photon's light speed $c_l$,
\beq
-3 \times 10^{-15} c_l < (c_g-c_l) < 7 \times 10^{-16} c_l,
\label{GRB}
\eeq
where the upper bound, or the lower bound, is obtained by assuming the simultaneous emission of the GW and GRB signals, or the GRB signals $10 s$ after the GW signal, respectively. But, not to mention the important change of the order of magnitudes by changing the models of photon emissions \cite{LIGO:2017}, we note that one can not tell about $c_g$ separately from the data (\ref{GRB}), without knowing about $c_l$ together, since (\ref{GRB}) is only about theirs difference $c_g-c_l$. The smallness of the relative difference $(c_g-c_l)/c_l$ in (\ref{GRB}) can be due to the smallness of $(\De c_g-\De c_l)/c_l$ for the modified speeds of gravity and light, $c_g=c+\De c_g, ~c_l=c+\De c_l,$ from the standard light speed $c$. And also, the observed time delay of the light can be due to the space-time curvature along its path.

In the literatures \cite{Emir:2017,Gong:2018}, on the other hand, it has always been assumed that $\De c_l=0$ and so (\ref{GRB}) has been considered as the absolute bound of $-3 \times 10^{-15}< \De c_g/c < 7 \times 10^{-16}$, which would be strong enough to exclude the modified speed of gravity $\De c_g$. However, if there is $\De c_l$ also, the story is completely different and the constraint (\ref{GRB}) yields $-3 \times 10^{-15}< (\De c_g-\De c_l)/c < 7 \times 10^{-16}$, even if $\De c_g$ and $\De c_l$ are not small separately! Actually, our exact solution for charged rotating black holes implies that $c_g=c_l$, {\it i.e.}, $\De c_g=\De c_l$ for the small perturbations around the flat Minkowski background \cite{Park:2009}. In that case, the time delay will be due to either the emission-time difference of the GW and GRB signals, or due to the photon's mass, either intrinsic or curvature induced, or even the graviton's mass which has not been studied in our solutions. Hence, our LV gravity with $c_g \neq c$ can be still consistent with the recently observed time delay of the coincident GW and GRB signals from a binary neutron star merger, contrary to the claims in the literature, with some possible scenarios to explain the observed data.

Another important constraints are the parameterized post-Newtonian (PPN) parameters which measure the preferred-frame effects \cite{Blas:2011,Emir:2017}, with the current bounds \cite{Will:2014},
\beq
|\al_1|&=&|8 (\widehat{\xi}-\si/2)| < 4 \times 10^{-5},\label{alpha_1} \\
|\al_2|&=&\left|\left(\f{\widehat{\xi}-\si/2}{\widehat{\xi}-\si/2+1}\right)
\left(1+\f{2 (2 \widehat{\la}+1) (\widehat{\xi}-\si/2)}{\widehat{\la}} \right)\right| < \times 10^{-9},\label{alpha_2}
\eeq
where $\widehat{\xi}=\xi-1$ and $\widehat{\la}=\la-1$.
For the case of $\widehat{\xi}-\si/2=0$, which includes the GR limit of
$\widehat{\xi}=0, \si=0$, we obtain $\al_1=\al_2=0$ which trivially satisfies
the constraints (\ref{alpha_1}) and (\ref{alpha_2}), without any restriction on
$\widehat{\la}$. However, for $\widehat{\xi}-\si/2\neq 0$, the situation is quite
different since, in the constraint (\ref{alpha_2}), $\widehat{\la}$ can not
contain the GR case of $\widehat{\la}=\la-1=0$ because it makes $\al_2\ra \infty$.
But it rather has a gap
% around
$\widehat{\la} \approx -2 (\widehat{\xi}-\si/2)+\widehat{\ep}$ with $\widehat{\ep}>2 \times 10^{-4}$ using $2 \widehat{\la} +1 \approx 1$ for a small $\widehat{\la}$, with the constraints $|\widehat{\xi}-\si/2|< 5 \times 10^{-6}$ from (\ref{alpha_1}). In other words, the LV parameters $\widehat{\xi}-\si/2$ and $\widehat{\la}$ are strongly correlated in \Ho~ gravity.

\section{Concluding remarks}

In conclusion, we have studied a procedure for finding rotating black hole solutions in the low-energy sector of the four-dimensional (non-projectable) \Ho~gravity as a viable LV gravity, and obtain the Kerr-type rotating black hole solutions with or without cosmological and electromagnetic charges.

It would be interesting to test the solution in the astrophysical observations. Applying our procedure to the renormalizable $(z=3)$ \Ho~gravity in four dimensions and finding the exact rotating black hole solutions will a challenging problem. The uniqueness of our solutions and their stabilities are remaining problems. Especially, it would be interesting to see the instability of {\it ultra-spinning} black holes $J/aM=1/\sqrt{\ka \xi}>1$ with $\ka \xi < 1$, similar to ultra-spinning Myers-Perry's black holes in GR \cite{Empa:2003}.\\

{\it Note added}: After finishing this paper, a related paper, which corresponds to $\ka \xi=1$, $\la \neq 1$ in our context, appeared in Einstein-aether gravity \cite{Fran:2023}, where the numerical rotating solutions have been found recently \cite{Adam:2021}. But the possibilities of our rotating solutions with $\ka \xi \neq 1$ were not considered due to the recent gravity speed constraints \cite{LIGO:2017} without considering the possibility of modified light speed \cite{Emir:2017,Gong:2018}.

\section*{Acknowledgments}

We would like to thank Ozgur Sarioglu for helpful correspondences.
This work was supported by Basic Science Research Program through the National
Research Foundation of Korea (NRF) funded by the Ministry of Education,
Science and Technology {(2020R1A2C1010372, 2020R1A6A1A03047877)}.

\appendix

\section{Explicit expressions of $\widehat{\rho}$ and $\widehat{p_a}$ in
${T^{\m}}_{\n (\it{eff})}$
%the effective energy-momentum tensor components
}

The explicit expressions of $\widehat{\rho}$ and $\widehat{p_a}$ in
the effective energy-momentum tensor ${T^{\m}}_{\n (\it{eff})}$ (\ref{T_eff}) are given as follows:
\beq
\widehat{\rho} &=& \widehat{\kappa\xi}~
%(\kappa  \xi-1 )
a^2 m^2  \text{sin}^2\theta
 \left\{
(a^4-2 a^2 r^2-3 r^4)^2-2 a^2 (a^6-5 a^4 r^2+a^2 (4 m-3 r)
r^3+3 r^6) \text{sin}^2\theta
\right.  \no \\
&&\left.
+a^4 (a^4-6 a^2 r^2+(8 m-3 r) r^3) \text{sin}^4\theta
 \right\}/Z, \\
\widehat{p_1} &=& \widehat{\kappa\xi} ~
%(\kappa  \xi-1 )
a^2 m^2  \text{sin}^2\theta \left\{(a^4-2 a^2 r^2-3 r^4)^2-2 a^2 (a^6-a^4 r^2+3 r^6+a^2 r^3 (-4
m+r)) \text{sin}^2\theta
\right.  \no \\
&&\left.
+a^4 (a^4+2 a^2 r^2+(-8 m+5 r)r^3 ) \text{sin}^4\theta\right\}/Z ,\\
\widehat{p_2} &=&2 \widehat{\kappa\xi} ~
%(\kappa  \xi-1 )
a^4 m^2 r \text{cos}\theta (a^2+r (-2 m+r)) \left[a^4-3 a^2 r^2-6 r^4+a^2 (a^2-r^2) \text{cos}(2 \theta)\right] \text{sin}^3\theta/Z, \\
\widehat{p_3} &=&2 \widehat{\kappa\xi}~a^4 m^2 r
%(-1+\kappa  \xi )
\text{cos}\theta \left[a^4-3 a^2 r^2-6 r^4+a^2 (a^2-r^2) \text{cos}(2 \theta )\right]
\text{sin}^3\theta/Z, \\
\widehat{p_4} &=&8 \sqrt{\kappa \xi} ~\widehat{\rho}~ {\it a} m r/\Si^2 ,\\
%-\widehat{\kappa\xi}\sqrt{\kappa \xi}~8 a^3 m^3 r
%(-1+\kappa  \xi )
%\text{sin}^2\theta \left\{(a^4-2 a^2 r^2-3 r^4)^2-2
%a^2 (a^6-5 a^4 r^2+a^2 (4 m-3 r) r^3+3 r^6) \text{sin}^2\theta
%\right.  \no \\
%&&\left.
%+a^4 (a^4-6 a^2 r^2+(8 m-3 r) r^3) \text{sin}^4\theta \right\}/Z, \\
Z&=&(a^2+r^2-a^2 \text{sin}^2\theta)^3 \left\{(a^2+r^2)^2-a^2 (a^2+r (-2 m+r)) \text{sin}^2\theta\right\}^2,
\eeq
where $\widehat{\kappa\xi} =\kappa  \xi-1$ is a LV
%Lorentz-violation
factor.

\section{The uniqueness of the invariant line element $ds^2$ under ${\it Diff}_{\cal F}$.}

In
%this Appendix we
order to prove the {\it uniqueness} of the invariant line element (\ref{metric}) under
%reduced \diff~
${\it Diff}_{\cal F}$, we start by expanding (\ref{metric}) as
\beq
ds^{2}=(-N^{2}+g_{ij}N^{i}N^{j})dt^{2}+2 g_{ij}N^{i}dx^{j}dt+g_{ij}dx^{i}dx^{j}.
\eeq
Then, under ${\it Diff}_{\cal F}$ (\ref{delg3}), the first term transforms as
\beq
\delta_{\xi}\left[ (-N^{2}+g_{ij}N^{i}N^{j})dt^{2}\right]&=&(\xi^{0}\partial_{0}g_{tt}+\xi^{i}
%\partial_0
\nabla_i g_{tt}+\underline{2N_{i}\partial_{0}\xi^{i}})dt^{2},\label{D1}
\eeq
and the second and third terms transform, respectively, as
\beq
\delta_{\xi} \left[2 g_{ij}N^{i}dx^{j}dt\right]&=& -\underline{2N_{j}\partial_0\xi^{j}}dt^{2}+(2\xi^{0}\partial_0N_{j}+2\xi^{i}\nabla_{i}N_{j}
+2N^{i}\nabla_{j}\xi_{i} \no \\
&&-2N_{k}\partial_{j}\xi^{k}+\underline{\underline{2g_{ij}\partial_0\xi^{i}}})dtdx^{j},\label{D2}\\
\delta_{\xi} \left[g_{ij}dx^{i}dx^{j}\right]&=&(2\nabla_{i}\xi_{j}+\xi^{0}\partial_0g_{ij}
-2g_{ik}\partial_{j}\xi^{k})dx^{i}dx^{j}-\underline{\underline{2g_{ij}\partial_0\xi^{j}}}dx^{i}dt.\label{D3}
\eeq
Summing up (\ref{D1}-\ref{D3}) and using the metric compatibility to change the covariant derivatives to the partial ones,
we have the form of Lie derivative of a scalar as follows:
%expected
\beq
\delta_{\xi}(ds^{2})&=&(\xi^{0}\partial_0g_{tt}+\xi^{i}\partial_{i}g_{tt})dt^{2}
+(2\xi^{0}\partial_0N_{j}+2\xi^{i}\partial_{i}N_{j})dt dx^{j} \nonumber\\
&&+(\xi^{0}\partial_0g_{ij}+\xi^{k}\partial_{k}g_{ij})dx^{i}dx^{j}\\
&=&{}\xi^{\mu} \partial_{\mu} (ds^{2}).
\eeq
Here, it is important to note that the {\it underlined} terms are
{\it exactly canceled} only for the line element (\ref{metric}), which means
the unique line element that is invariant under ${\it Diff}_{\cal F}$ (\ref{delg3}).
This is quite remarkable since it seems to be contradict to the transformation
of action (\ref{horava}), in which each term is invariant {\it separately}
under ${\it Diff}_{\cal F}$ so that one can introduce arbitrary coupling
constant for each term as the LV
%Lorentz-violation
effect. The basic reason is that $dx^i$ is $not$ a spacial projection of $dx^{\mu}$, but rather it can transform to $dt$ also (though not for the reverse) when $\partial_0 \xi^i \neq 0$, contrary to $K_{ij}, R_{ij}, \nabla_i, etc$ in the action (\ref{horava}). This is a quite powerful property of our LV gravity that enables us to study invariant properties of particles via $ds^{2}$ as in GR, contrary to widespread beliefs. Our proof, which has never been known in literature, justifies the use of $ds^{2}$ even in \Ho~gravity case also, even though the full {\it Diff} is apparently broken.

\section{Petrov classification, Killing tensor, and Hamilton-Jacobi equation}

For the Petrov classification of our rotating solution (\ref{massive_Kerr_metric}) without cosmological constant, we first find the four {\it null} vectors,
\beq
\widehat{l}^{\m}&=&\f{1}{\De_r} \left[r^2+a^2,\De_r,0,a+ (r^2+a^2) {\it H}\right], \no \\
\widehat{n}^{\m}&=&\f{1}{2 \rho^2} \left[r^2+a^2,-\De_r,0,a+ (r^2+a^2) {\it H}\right], \no \\
\widehat{m}^{\m} &=&\f{1}{\sqrt{2} (r+i {\it a} \mbox{cos}\theta)} \left[i {\it a } \mbox{sin}\theta,0,1,i+i{\it a} \mbox{sin}\theta {\it H} \right],
\label{null_new}
\eeq
which satisfy the usual orthogonality conditions $\widehat{l}^{\m} \widehat{n}_{\m}=-1, \widehat{m}^{\m} \widehat{\bar{m}}_{\m}=1$ with ${\it H}=(\sqrt{\ka \xi}-1) 2mra/\Si^2$, and our metric can be written as $g_{\m \n}=2 \widehat{m}_{(\m} \widehat{\bar{m}}_{\n)}-2 \widehat{l}_{(\m} \widehat{n}_{\n)}$, as usual. But, we find that there are {\it four} distinct principal null directions $k^{\m}$
that satisfy $k^{\al}k^{\be}k_{[\m} C_{\n ] \al \be [\rho} k_{\si]}=0$ with the Weyl tensor $C_{\n \al \be \rho}$, such that our solution is the Petrov type I \footnote{Similar results for the perturbed spacetimes are also obtained recently \cite{Aran:2015,Owen:2021}.}. This is contrary to Kerr spacetime in GR, where two null vectors $l^{\m}$ and $n^{\n}$ are aligned with the principal null direction and it is the Petrov type D.

%As will be discussed briefly below,
Since our rotating solution (\ref{massive_Kerr_metric}) is the Petrov type I,
%with four distinct null vectors, contrary to the Petrov type D for Kerr solution,
%and so
several well-established and nice properties of type D are not guaranteed. For example, the existence of Killing tensor and also the separability of Hamilton-Jacobi (HJ) equation are not guaranteed \cite{Cart:1973,Walk:1970}. Especially for Carter's proof, our solution can {\it not} be expressed as the generic form of solutions which generates the Killing tensor and separable HJ equation \cite{Cart:1973}.

Actually one can check that the {\it used-to-be} Killing tensor form of Kerr metric $L_{\m \n}=2 \rho^2 m_{(\m} \bar{m}_{\n)}-a^2 \mbox{cos}^2 \theta g^{(4)}_{\m \n}$ with the complex null vectors $m_{\m}, \bar{m}_{\n}$ of the Newman-Penrose basis in GR, nor its correspondent in our new null vectors (\ref{null_new}) $\widehat{L}_{\m \n}=2 \rho^2 \widehat{m}_{(\m} \widehat{\bar{m}}_{\n)}-a^2 \mbox{cos}^2 \theta g^{(4)}_{\m \n}$, do not satisfy the Killing tensor equation $\nabla^{(4)}_{(\m} L^{}_{\n \si)}=0$ or $\nabla^{(4)}_{(\m} \widehat{L}^{}_{\n \si)}=0$. However, interestingly, we can find a certain geodesic with tangent vector $v^{\m}$ such that $v^{\n} v^{\si}L_{\n \si}$ is still {\it constant} along the geodesics, {\it i.e.},
\beq
v^{\m} \nabla^{(4)}_{\m} (v^{\n}v^{\si} L_{\n \si})=-\f{24 (\ka \xi-1) (mra^2)^2~ v_{\theta} v_{\phi}^2~ \mbox{sin}\theta \mbox{cos}\theta}{\Si^4}=0.
\label{Killing_condition}
\eeq
The fixed-angle geodesics, $v_{\theta}=0$ ($\theta$=const.) or $v_{\phi}=0$ ($\phi$=const.) are those cases and this supports our previous analysis of $\theta=\pi/2$ plane in Sec. VI. For other more general geodesics with $v_{\theta}v_{\phi} \neq 0$, we may need numerical analysis to integrate the geodesics if there is no exact Killing tensor \footnote{If one considers successive applications of infinitesimal geodesics with fixed-angle planes, we may produce  general geodesics in the continuous limit. The condition (\ref{Killing_condition}), then, can be generalized into an integral form $\int_{\ga}v^{\m} \nabla^{(4)}_{\m} (v^{\n}v^{\si} L_{\n \si})=0$ along a general geodesic $\ga$.}.

A related difficulty in integrating the full geodesics is the {\it lack of separability} of HJ equation,
\beq
-2 \f{\pa S}{\pa \tau}&=&g^{(4)\m \n} \pa_{\m}S \pa_{\n}S \no \\
&=&-\f{\Si^2}{\rho^2 \De_r}(\pa_t S)^2+\f{4  mra \sqrt{\ka \xi}}{\rho^2 \De_r}\pa_t S \pa_{\phi}S+\f{\De_r \rho^4 - (2mra)^2  \ka \xi \mbox{sin}^2\theta}{\Si^2 \rho^2 \De_r \mbox{sin}^2\theta}(\pa_{\phi}S)^2 \no \\
&&+\f{\De_r}{\rho^2}(\pa_r S)^2+\f{1}{\rho^2}(\pa_{\theta} S)^2
\label{HJ}
\eeq
from a particle Hamiltonian $H=(1/2)g^{\m\n}p_{\m} p_{\n}$ and $p_{\m}=\pa_{\m}S$ with the Hamilton's principal function $S$. By ``assuming" the separability, we consider \cite{Chan:1985}
\beq
S=\f{1}{2} \m^2 \tau -Et+L \phi+S_r (r)+S_{\theta} (\theta)
\eeq
then, we finally obtain the master equation
\beq
\left\{
\m^2 r^2 +\left(a E+\sqrt{\ka \xi} L \right)^2-\f{1}{\De_r} \left((r^2+a^2)E +\sqrt{\ka \xi} a L \right)^2+\De_r \left( \f{d S_r}{dr}\right)^2
+(\ka\xi-1) \f{a^2 L^2}{\De_r}
\right\} \no \\
+\left\{
\m^2 a^2 \mbox{cos}^2\theta+ \left(-a^2 E^2+\f{\ka \xi L^2}{\mbox{sin}^2\theta} \right)
+\left( \f{d S_{\theta}}{d\theta} \right)^2 -(\ka\xi-1) \f{L^2}{\mbox{sin}^2\theta}
\right\}
%\no \\&+&
+(\ka\xi-1) \f{(2mra)^2 L^2}{\Si^2 \De_r}=0,\no \\
\label{HJ_master}
\eeq
where $\m$ is the particle's rest mass. The last term in (\ref{HJ_master}) breaks the separability of HJ equation in our LV gravity with $\ka \xi \neq 1$, unless we consider $L=0$ or $\theta$=constant, similar to the above Killing tensor analysis.

\section{Equations for the Lorentz-violating Maxwell action}

In order to find the LV Maxwell action, we first consider
the covariant Maxwell action,
\beq
S_{M_0}=\eta\int d^{4}x \sqrt{-g^{(4)}}  F_{\mu\nu}F^{\mu\nu}
\label{covmax}
\eeq
with the electromagnetic coupling constant $\eta$.
Applying the usual ADM decomposition to this action and, due to ${\it Diff}_{\cal F}$ (\ref{delg3}), introducing
an arbitrary coupling
 %(note that we have introduced a breaking
 parameter $\zeta$ where the covariant case (\ref{covmax}) corresponds to $\zeta=\eta$, we obtain the LV Maxwell action,
\beq
S_{M}=\int_{{\bf R}\times \Si_t} dt d^{3}x \sqrt{g} N \left(\zeta F_{ij}F^{ij}\mp \frac{2\eta}{N^{2}}\left[E_{i}+F_{ij}N^{j}\right]^{2}\right),
\label{LV_Max-action}
\eeq
where $\mp$ corresponds to a normalization of the normal vector
$n_{\mu}n^{\mu}=\mp 1$ for a
%{\it time-slicing}
hypersurface $\Si_t$,
%of the decomposition i.e. ,
and $E_{i}\equiv
%\partial_{0}
\dot{A}_{i}-\partial_{i}A_{0},~ F_{ij}=\nabla_i A_j-\nabla_j A_i$.

The variations of the LV Maxwell action (\ref{LV_Max-action}) give, neglecting boundary terms and choosing the hypersurface $\Si_t$ as a time-foliation with $n_{\mu}n^{\mu}=-1$,
\beq
{\cal H}_M\equiv \frac{\delta S_{M}}{\delta N}&=&\zeta F_{ij}F^{ij}+2 \eta L_{i}L^{i},\\
{\cal H}^i_{M}\equiv \frac{\delta S_{M}}{\delta N_{i}}&=&4\eta L_{j}F^{ji}, \\
E_{ij}^M \equiv \frac{\delta S_{M}}{\delta g^{ij}}&=&-\frac{1}{2}N\mathcal{L}_M g_{ij}+2\zeta N F_{i}{}^{k}F_{kj}-4\eta  L^{k}F_{k(j}N_{i)}-2 \eta N L_{i}L_{j},\\
E_{0}^M \equiv \frac{\delta S_{M}}{\delta A_{0}}&=&\eta \nabla^{i}L_{i}=0,\\
E_{i}^M \equiv\frac{\delta S_{M}}{\delta A_{i}}&=& -\zeta \nabla_{j}(4N F^{ji})+4 \eta \dot L^{i}+4 \eta \nabla_{j}\left(L^{j}N^{i}-L^{i}N^{j}\right)=0,\\
\eeq
where $\mathcal{L}_M=\zeta F_{ij}F^{ij}-2 \eta L_{i}L^{i} $ and $L_{i}\equiv N^{-1} \left(E_{i}+F_{ij}N^{j}\right)$. Then, the total gravity equations are given by ${\cal H}_{tot}={\cal H}+{\cal H}_M=0,~ {\cal H}^i_{tot}={\cal H}^i+{\cal H}^i_{M}=0,$ and ${E_{ij}}^{tot}=E_{ij}+E_{ij}^M=0.$

%%%%%%%%%% References %%%%%%%%%%%%%%%%%%%%%%%%%
\newcommand{\J}[4]{#1 {\bf #2} #3 (#4)}
\newcommand{\andJ}[3]{{\bf #1} (#2) #3}
\newcommand{\AP}{Ann. Phys. (N.Y.)}
\newcommand{\MPL}{Mod. Phys. Lett.}
\newcommand{\NP}{Nucl. Phys.}
\newcommand{\PL}{Phys. Lett.}
\newcommand{\PR}{Phys. Rev. D}
\newcommand{\PRL}{Phys. Rev. Lett.}
\newcommand{\PTP}{Prog. Theor. Phys.}
\newcommand{\hep}[1]{ hep-th/{#1}}
\newcommand{\hepp}[1]{ hep-ph/{#1}}
\newcommand{\hepg}[1]{ gr-qc/{#1}}
\newcommand{\bi}{ \bibitem}
%%%%%%%%%%%%%%%%%%%%%%%%%%%%%%%%%%%%%%%%%%%%%%%

\begin{thebibliography}{999}

%\cite{Abbott:2016blz}
%\bibitem{Abbo}
 % B.~P.~Abbott {\it et al.} [LIGO Scientific and Virgo Collaborations],
  %``Observation of Gravitational Waves from a Binary Black Hole Merger,''
 % Phys.\ Rev.\ Lett.\  {\bf 116}, %no. 6,
 % 061102 (2016);
  %doi:10.1103/PhysRevLett.116.061102
 % [arXiv:1602.03837 [gr-qc]];

\bibitem{Kerr:1963}
  R.~P.~Kerr,
  %``Gravitational field of a spinning mass as an example of algebraically special metrics,''
  Phys.\ Rev.\ Lett.\  {\bf 11}, 237 (1963).
 % doi:10.1103/PhysRevLett.11.237

%\cite{Chandrasekhar:1985kt}
\bibitem{Chan:1985}
S.~Chandrasekhar,
``The mathematical theory of black holes,'' (Oxford Univ.
%ersity
Press, 1983).
%348 citations counted in INSPIRE as of 12 Jan 2024

%\cite{Teukolsky:2014vca}
%\bibitem{Teuk:2014}
%S.~A.~Teukolsky,
%``The Kerr Metric,''
%Class. Quant. Grav. \textbf{32} (2015),
% no.12,
%124006.
%doi:10.1088/0264-9381/32/12/124006
%[arXiv:1410.2130 [gr-qc]].
%141 citations counted in INSPIRE as of 12 Jan 2024

%\cite{Carter:1973rla}
\bibitem{Cart:1973}
B.~Carter, %``Black holes equilibrium states'' 
in {\it Les Astre Occlus}, Gordon and Breach, New York (1973).

%in {\it Les Houches Summer School of Theoretical Physics: {Black Holes}} (1973).
%10 citations counted in INSPIRE as of 12 Jan 2024

%\cite{Walker:1970un}
\bibitem{Walk:1970}
M.~Walker and R.~Penrose,
%``On quadratic first integrals of the geodesic equations for type [22] spacetimes,''
Commun. Math. Phys. \textbf{18}, 265
%-274
(1970).
%doi:10.1007/BF01649445
%353 citations counted in INSPIRE as of 12 Jan 2024

\bibitem{DeWi:1967}
  B.~S.~DeWitt,
  %``Quantum Theory of Gravity. 1. The Canonical Theory,''
  Phys.\ Rev.\  {\bf 160}, 1113 (1967).
 % doi:10.1103/PhysRev.160.1113

\bibitem{Hora:2009} P.~Ho\v{r}ava,
  %``Membranes at Quantum Criticality,''
  %JHEP {\bf 0903}, 020 (2009);
  %[arXiv:0812.4287 [hep-th]].
  %%CITATION = JHEPA,0903,020;%%
%\bibitem{Hora} P.~Ho\v{r}ava,
  %``Quantum Gravity at a Lifshitz Point,''
  Phys.\ Rev.\  D {\bf 79}, 084008 (2009).
  %[arXiv:0901.3775 [hep-th]].
%\cite{Park:2009gf}

\bibitem{Park:2009}
M.~I.~Park,
%``Remarks on the Scalar Graviton Decoupling and Consistency of Horava Gravity,''
Class. Quant. Grav. \textbf{28}, 015004 (2011).
%doi:10.1088/0264-9381/28/1/015004
%[arXiv:0910.1917 [hep-th]].
%43 citations counted in INSPIRE as of 19 Jan 2024

%\bibitem{Jack}

%\cite{Blas:2009qj}
\bibitem{Blas:2009}
D.~Blas, O.~Pujolas and S.~Sibiryakov,
%``Consistent Extension of Horava Gravity,''
Phys. Rev. Lett. \textbf{104}, 181302 (2010).
%doi:10.1103/PhysRevLett.104.181302
%[arXiv:0909.3525 [hep-th]].
%561 citations counted in INSPIRE as of 12 Jan 2024

%\cite{Jacobson:2013xta}
\bibitem{Jaco:2013}
T.~Jacobson,
%``Undoing the twist: The Ho\v{r}ava limit of Einstein-aether theory,''
Phys. Rev. D \textbf{89}, 081501 (2014).
%doi:10.1103/PhysRevD.89.081501
%[arXiv:1310.5115 [gr-qc]].
%90 citations counted in INSPIRE as of 12 Jan 2024

%\cite{Blas:2011zd}
\bibitem{Blas:2011}
D.~Blas and H.~Sanctuary,
%``Gravitational Radiation in Ho\v{r}ava Gravity,''
Phys. Rev. D \textbf{84}, 064004 (2011).
%doi:10.1103/PhysRevD.84.064004
%[arXiv:1105.5149 [gr-qc]].
%77 citations counted in INSPIRE as of 12 Jan 2024

%\cite{Blas:2011ni}
%\bibitem{Blas:2011}
%D.~Blas and S.~Sibiryakov,
%``Horava gravity versus thermodynamics: The Black hole case,''
%Phys. Rev. D \textbf{84}, 124043 (2011).
%doi:10.1103/PhysRevD.84.124043
%[arXiv:1110.2195 [hep-th]].
%151 citations counted in INSPIRE as of 12 Jan 2024

%\cite{Park:2023byp}
\bibitem{Park:2023}
M.~I.~Park and H.~W.~Lee,
%``Massless Rotating Spacetimes in Four-Dimensional Horava Gravity,''
arXiv:2309.13859 [hep-th].
%0 citations counted in INSPIRE as of 12 Jan 2024

%\cite{Boyer:1966qh}
%\bibitem{Boye:1966}
%R.~H.~Boyer and R.~W.~Lindquist,
%``Maximal analytic extension of the Kerr metric,''
%J. Math. Phys. \textbf{8}, 265 (1967).
%doi:10.1063/1.1705193
%675 citations counted in INSPIRE as of 18 Sep 2023

%\cite{Park:2001zn}
\bibitem{Park:2001}
M.~I.~Park,
%``Hamiltonian dynamics of bounded space-time and black hole entropy: Canonical method,''
Nucl. Phys. B \textbf{634}, 339
%-369
(2002).
%doi:10.1016/S0550-3213(02)00292-4
%[arXiv:hep-th/0111224 [hep-th]].
%102 citations counted in INSPIRE as of 12 Aug 2023

%\cite{Gibbons:1977mu}
\bibitem{Gibb:1977}
G.~W.~Gibbons and S.~W.~Hawking,
%``Cosmological Event Horizons, Thermodynamics, and Particle Creation,''
Phys. Rev. D \textbf{15}, 2738
%-2751
(1977).
%doi:10.1103/PhysRevD.15.2738
%2894 citations counted in INSPIRE as of 12 Jan 2024

%\cite{Gibbons:2017djb}
\bibitem{Gibb:2017}
G.~W.~Gibbons and M.~S.~Volkov,
%``Zero mass limit of Kerr spacetime is a wormhole,''
Phys. Rev. D \textbf{96},
%no.2,
024053 (2017).
%doi:10.1103/PhysRevD.96.024053
%[arXiv:1705.07787 [hep-th]].
%25 citations counted in INSPIRE as of 16 Aug 2023

%\cite{Banados:1992gq}
\bibitem{Bana:1992}
M.~Banados \textit{et al.},
%M.~Henneaux, C.~Teitelboim and J.~Zanelli,
%``Geometry of the (2+1) black hole,''
Phys. Rev. D \textbf{48}, 1506%-1525
(1993).
%[erratum: Phys. Rev. D \textbf{88}, 069902 (2013)].
%doi:10.1103/PhysRevD.48.1506
%[arXiv:gr-qc/9302012 [gr-qc]].
%1813 citations counted in INSPIRE as of 12 Jan 2024

%\cite{Devecioglu:2021zfb}
\bibitem{Deve:2021}
D.~O.~Devecioglu and M.~I.~Park,
%``Symmetries and conservation laws in Ho\v{r}ava gravity,''
Phys. Rev. D \textbf{108}, %no.6,
064030 (2023).
%doi:10.1103/PhysRevD.108.064030
%[arXiv:2112.00576 [hep-th]].
%2 citations counted in INSPIRE as of 12 Jan 2024

\bibitem{Lu:2009}
  H.~Lu, J.~Mei and C.~N.~Pope,
  %``Solutions to Horava Gravity,''
Phys.\ Rev.\ Lett.\  {\bf 103}, 091301 (2009);
  %[arXiv:0904.1595 [hep-th]].
  %%CITATION = ARXIV:0904.1595;%%
%\bibitem{Keha}
  A.~Kehagias and K.~Sfetsos,
  %``The black hole and FRW geometries of non-relativistic gravity,''
Phys.\ Lett.\ B {\bf 678}, 123 (2009);
  %[arXiv:0905.0477 [hep-th]].
 % \bibitem{Park:0905}
  M.~I.~Park,
  %``The Black Hole and Cosmological Solutions in IR modified Ho\v{r}ava Gravity,''
  JHEP {\bf 0909}, 123 (2009);
  %  [arXiv:0905.4480 [hep-th]].
%\cite{Kiritsis:2009rx}
%\bibitem{Kiritsis:2009rx}
E.~B.~Kiritsis and G.~Kofinas,
%``On Horava-Lifshitz 'Black Holes',''
JHEP \textbf{01}, 122 (2010).
%doi:10.1007/JHEP01(2010)122
%[arXiv:0910.5487 [hep-th]].
%123 citations counted in INSPIRE as of 12 Jan 2024

%\cite{Park:2012ev}
\bibitem{Park:2012}
M.~I.~Park,
%``The rotating black hole in renormalizable quantum gravity: The three-dimensional Ho$\check r$ava gravity case,''
Phys. Lett. B \textbf{718}, 1137
%-1141
(2013) [Erratum: Phys. Lett. B \textbf{809}, 135720 (2020)];
%doi:10.1016/j.physletb.2012.11.067
%[arXiv:1207.4073 [hep-th]].
%12 citations counted in INSPIRE as of 12 Jan 2024
%\cite{Park:2020yis}
%\bibitem{Park:2020yis}
%M.~I.~Park,
%``Rotating black holes in 3D Ho\v{r}ava gravity revisited,''
PTEP \textbf{2022}, %no.11,
113E02 (2022);
%doi:10.1093/ptep/ptac147
%[arXiv:2008.06574 [hep-th]].
%0 citations counted in INSPIRE as of 12 Jan 2024
%\cite{Sotiriou:2014gna}
%\bibitem{Sotiriou:2014gna}
T.~P.~Sotiriou, I.~Vega and D.~Vernieri,
%``Rotating black holes in three-dimensional Ho\v{r}ava gravity,''
Phys. Rev. D \textbf{90}, %no.4,
044046 (2014).
%doi:10.1103/PhysRevD.90.044046
%[arXiv:1405.3715 [gr-qc]].
%57 citations counted in INSPIRE as of 12 Jan 2024

%\cite{Gibbons:2004uw}
\bibitem{Gibb:2004}
G.~W.~Gibbons \textit{et al.},
%H.~Lu, D.~N.~Page and C.~N.~Pope,
%``The General Kerr-de Sitter metrics in all dimensions,''
J. Geom. Phys. \textbf{53}, 49
%-73
(2005), Appendix {\bf E}.
%doi:10.1016/j.geomphys.2004.05.001
%[arXiv:hep-th/0404008 [hep-th]].
%424 citations counted in INSPIRE as of 12 Jan 2024

\bibitem{Curi:1979}
A.~Curir,
%Spinentropyofarotatingblackhole,
Il Nuovo Cimento {\bf B52}, 262
%-266.
(1979).

%\cite{Klemm:2004mb}
\bibitem{Klemm:2004}
D.~Klemm and L.~Vanzo,
%``Aspects of quantum gravity in de Sitter spaces,''
JCAP \textbf{11}, 006 (2004).
%doi:10.1088/1475-7516/2004/11/006
%[arXiv:hep-th/0407255 [hep-th]].
%56 citations counted in INSPIRE as of 26 Jan 2024

%\cite{Park:2006hu}
\bibitem{Park:2006}
M.~I.~Park,
%``Thermodynamics of exotic black holes, negative temperature, and Bekenstein-Hawking entropy,''
Phys. Lett. B \textbf{647}, 472
%-476
(2007);
%doi:10.1016/j.physletb.2007.02.036
%[arXiv:hep-th/0602114 [hep-th]].
%30 citations counted in INSPIRE as of 26 Jan 2024
%\cite{Park:2006pb}
%\cite{Park:2006fp}
%\bibitem{Park:2006fp}
%M.~I.~Park,
%``Can Hawking temperatures be negative?,''
Phys. Lett. B \textbf{663}, 259
%-264
(2008);
%doi:10.1016/j.physletb.2008.04.009
%[arXiv:hep-th/0610140 [hep-th]].
%25 citations counted in INSPIRE as of 26 Jan 2024
%\bibitem{Park:2006}
%M.~I.~Park,
%``Thoughts on the Area Theorem,''
Class. Quant. Grav. \textbf{25}, 095013 (2008).
%doi:10.1088/0264-9381/25/9/095013
%[arXiv:hep-th/0611048 [hep-th]].
%15 citations counted in INSPIRE as of 26 Jan 2024

%\cite{Cvetic:2018dqf}
\bibitem{Cveti:2018}
M.~Cveti\v{c} \textit{et al.},
% G.~W.~Gibbons, H.~L\"u and C.~N.~Pope,
%``Killing Horizons: Negative Temperatures and Entropy Super-Additivity,''
Phys. Rev. D \textbf{98},
%no.10,
106015 (2018);
%doi:10.1103/PhysRevD.98.106015
%[arXiv:1806.11134 [hep-th]].
%28 citations counted in INSPIRE as of 26 Jan 2024
%\cite{Jacobson:2018ahi}
%\bibitem{Jacobson:2018ahi}
T.~Jacobson and M.~Visser,
%``Gravitational Thermodynamics of Causal Diamonds in (A)dS,''
SciPost Phys. \textbf{7},
%no.6,
079 (2019).
%doi:10.21468/SciPostPhys.7.6.079
%[arXiv:1812.01596 [hep-th]].
%65 citations counted in INSPIRE as of 26 Jan 2024

%\cite{Hawking:1998kw}
\bibitem{Hawk:1998}
S.~W.~Hawking, C.~J.~Hunter and M.~Taylor,
%``Rotation and the AdS / CFT correspondence,''
Phys. Rev. D \textbf{59}, 064005 (1999).
%doi:10.1103/PhysRevD.59.064005
%[arXiv:hep-th/9811056 [hep-th]].
%584 citations counted in INSPIRE as of 12 Jan 2024

\bibitem{Henn:1985}
  M.~Henneaux and C.~Teitelboim,
  %``Asymptotically anti-De Sitter Spaces,''
  Commun.\ Math.\ Phys.\  {\bf 98}, 391 (1985).
 % doi:10.1007/BF01205790

%\cite{Caldarelli:1999xj}
\bibitem{Cald:1999}
M.~M.~Caldarelli, G.~Cognola and D.~Klemm,
%``Thermodynamics of Kerr-Newman-AdS black holes and conformal field theories,''
Class. Quant. Grav. \textbf{17}, 399
%-420
(2000).
%doi:10.1088/0264-9381/17/2/310
%[arXiv:hep-th/9908022 [hep-th]].
%776 citations counted in INSPIRE as of 26 Jan 2024

%\cite{Gibbons:2004ai}
\bibitem{Gibb:2004b}
G.~W.~Gibbons, M.~J.~Perry and C.~N.~Pope,
%``The First law of thermodynamics for Kerr-anti-de Sitter black holes,''
Class. Quant. Grav. \textbf{22}, 1503
%-1526
(2005).
%doi:10.1088/0264-9381/22/9/002
%[arXiv:hep-th/0408217 [hep-th]].
%437 citations counted in INSPIRE as of 12 Jan 2024

%\bibitem{Klem:1998}
%  D.~Klemm, V.~Moretti and L.~Vanzo,
  %``Rotating topological black holes,''
%  Phys.\ Rev.\ D {\bf 57}, 6127 (1998)
%  [erratum: Phys.\ Rev.\ D {\bf 60}, 109902 (1999)].
 % doi:10.1103/PhysRevD.60.109902, 10.1103/PhysRevD.57.6127
 % [gr-qc/9710123].

%\cite{Sudarsky:1992ty}
\bibitem{Suda:1992}
D.~Sudarsky and R.~M.~Wald,
%``Extrema of mass, stationarity, and staticity, and solutions to the Einstein Yang-Mills equations,''
Phys. Rev. D \textbf{46}, 1453
%-1474
(1992).
%doi:10.1103/PhysRevD.46.1453
%242 citations counted in INSPIRE as of 19 Jan 2024

\bibitem{Kiri:2009}
 E.~Kiritsis and G.~Kofinas,
  %``Horava-Lifshitz Cosmology,''
  Nucl.\ Phys.\  B {\bf 821} (2009) 467.
 % [arXiv:0904.1334 [hep-th]].
  %%CITATION = NUPHA,B821,467;%%

%\cite{Lin:2014ija}
\bibitem{Lin:2014}
K.~Lin \textit{et al.},
%, E.~Abdalla, R.~G.~Cai and A.~Wang,
%``Universal horizons and black holes in gravitational theories with broken Lorentz symmetry,''
Int. J. Mod. Phys. D \textbf{23},
%no.13,
1443004 (2014).
%doi:10.1142/S0218271814430044
%[arXiv:1408.5976 [gr-qc]].
%39 citations counted in INSPIRE as of 19 Jan 2024

%\cite{Restuccia:2019xdo}
\bibitem{Rest:2019}
A.~Restuccia and F.~Tello-Ortiz,
%``Pure electromagnetic-gravitational interaction in Ho\v{r}ava\textendash{}Lifshitz theory at the kinetic conformal point,''
Eur. Phys. J. C \textbf{80},
%no.2,
86 (2020).
%doi:10.1140/epjc/s10052-020-7674-7
%[arXiv:1908.06581 [hep-th]].
%9 citations counted in INSPIRE as of 19 Jan 2024

%\cite{Hawking:1971tu}
\bibitem{Hawk:1971}
S.~W.~Hawking,
%``Gravitational radiation from colliding black holes,''
Phys. Rev. Lett. \textbf{26}, 1344
%-1346
(1971).
%doi:10.1103/PhysRevLett.26.1344
%998 citations counted in INSPIRE as of 12 Jan 2024

\bibitem{Penr:1969} R. Penrose, Riv. Nuovo Cimento {\bf 1},
252 (1969).

%\cite{LIGOScientific:2017zic}
\bibitem{LIGO:2017}
B.~P.~Abbott \textit{et al.},
% [LIGO Scientific, Virgo, Fermi-GBM and INTEGRAL],
%``Gravitational Waves and Gamma-rays from a Binary Neutron Star Merger: GW170817 and GRB 170817A,''
Astrophys. J. Lett. \textbf{848},
%no.2,
L13 (2017).
%doi:10.3847/2041-8213/aa920c
%[arXiv:1710.05834 [astro-ph.HE]].
%2766 citations counted in INSPIRE as of 12 Jan 2024

%\cite{EmirGumrukcuoglu:2017cfa}
\bibitem{Emir:2017}
A.~E.
%mir
G\"umr\"uk\c{c}\"uo\u{g}lu, M.~Saravani and T.~P.~Sotiriou,
%``Ho\v{r}ava gravity after GW170817,''
Phys. Rev. D \textbf{97},
%no.2,
024032 (2018).
%doi:10.1103/PhysRevD.97.024032
%[arXiv:1711.08845 [gr-qc]].
%115 citations counted in INSPIRE as of 12 Jan 2024

%\cite{Gong:2018vbo}
\bibitem{Gong:2018}
Y.~Gong \textit{et al.},
%S.~Hou, E.~Papantonopoulos and D.~Tzortzis,
%``Gravitational waves and the polarizations in Ho\v{r}ava gravity after GW170817,''
Phys. Rev. D \textbf{98},
%no.10,
104017 (2018).
%doi:10.1103/PhysRevD.98.104017
%[arXiv:1808.00632 [gr-qc]].
%34 citations counted in INSPIRE as of 12 Jan 2024

%\cite{Will:2014kxa}
\bibitem{Will:2014}
C.~M.~Will,
%``The Confrontation between General Relativity and Experiment,''
Living Rev. Rel. \textbf{17}, 4 (2014).
%doi:10.12942/lrr-2014-4
%[arXiv:1403.7377 [gr-qc]].
%2191 citations counted in INSPIRE as of 12 Jan 2024

%\cite{Emparan:2003sy}
\bibitem{Empa:2003}
R.~Emparan and R.~C.~Myers,
%``Instability of ultra-spinning black holes,''
JHEP \textbf{09}, 025 (2003).
%doi:10.1088/1126-6708/2003/09/025
%[arXiv:hep-th/0308056 [hep-th]].
%255 citations counted in INSPIRE as of 12 Jan 2024

%\cite{Franzin:2023rdl}
\bibitem{Fran:2023}
E.~Franzin, S.~Liberati and J.~Mazza,
%``A Kerr Black Hole in Einstein--\AE{}ther Gravity,''
arXiv:2312.06891 [gr-qc].
%0 citations counted in INSPIRE as of 12 Jan 2024

%\cite{Adam:2021vsk}
\bibitem{Adam:2021}
A.~Adam \textit{et al.},
%P.~Figueras, T.~Jacobson and T.~Wiseman,
%``Rotating black holes in Einstein-aether theory,''
Class. Quant. Grav. \textbf{39},
%no.12,
125001 (2022).
%doi:10.1088/1361-6382/ac5053
%[arXiv:2108.00005 [gr-qc]].
%20 citations counted in INSPIRE as of 12 Jan 2024

%\cite{Araneda:2015gsa}
\bibitem{Aran:2015}
B.~Araneda and G.~Dotti,
%``Petrov type of linearly perturbed type D spacetimes,''
Class. Quant. Grav. \textbf{32},
%no.19,
195013 (2015).
%doi:10.1088/0264-9381/32/19/195013
%[arXiv:1502.07153 [gr-qc]].
%11 citations counted in INSPIRE as of 12 Jan 2024

%\cite{Owen:2021eez}
\bibitem{Owen:2021}
C.~B.~Owen, N.~Yunes and H.~Witek,
%``Petrov type, principal null directions, and Killing tensors of slowly rotating black holes in quadratic gravity,''
Phys. Rev. D \textbf{103}, %no.12,
124057 (2021).
%doi:10.1103/PhysRevD.103.124057
%[arXiv:2103.15891 [gr-qc]].
%16 citations counted in INSPIRE as of 12 Jan 2024


%\cite{LIGOScientific:2021sio}
%\bibitem{LIGO:2021}
%R.~Abbott \textit{et al.},
% [LIGO Scientific, VIRGO and KAGRA],
%``Tests of General Relativity with GWTC-3,''
%[arXiv:2112.06861 [gr-qc]].
%298 citations counted in INSPIRE as of 08 Aug 2023


\end{thebibliography}

\end{document}
